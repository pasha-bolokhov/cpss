\documentclass[epsfig,12pt]{article}
\usepackage{epsfig}
\usepackage{graphicx}
\usepackage{rotating}
\usepackage{latexsym}
\usepackage{amsmath}
\usepackage{amssymb}
\usepackage{relsize}
\usepackage{geometry}
\geometry{letterpaper}
\usepackage{color}
\usepackage{bm}
\usepackage{slashed}
\usepackage{showlabels}




%%%%%%%%%%%%%%%%%%%%%%%%%%%%%%%%%%%%%%%%%%%%%%%%%%%%%%%%%%%%%%%%%%%%%%%%%%%%%%%%
%                                                                              %
%                                                                              %
%                     D O C U M E N T   S E T T I N G S                        %
%                                                                              %
%                                                                              %
%%%%%%%%%%%%%%%%%%%%%%%%%%%%%%%%%%%%%%%%%%%%%%%%%%%%%%%%%%%%%%%%%%%%%%%%%%%%%%%%
\def\baselinestretch{1.1}
\renewcommand{\theequation}{\thesection.\arabic{equation}}

\hyphenation{con-fi-ning}
\hyphenation{Cou-lomb}
\hyphenation{Yan-ki-e-lo-wicz}




%%%%%%%%%%%%%%%%%%%%%%%%%%%%%%%%%%%%%%%%%%%%%%%%%%%%%%%%%%%%%%%%%%%%%%%%%%%%%%%%
%                                                                              %
%                                                                              %
%                      C O M M O N   D E F I N I T I O N S                     %
%                                                                              %
%                                                                              %
%%%%%%%%%%%%%%%%%%%%%%%%%%%%%%%%%%%%%%%%%%%%%%%%%%%%%%%%%%%%%%%%%%%%%%%%%%%%%%%%
\def\beq{\begin{equation}}
\def\eeq{\end{equation}}
\def\beqn{\begin{eqnarray}}
\def\eeqn{\end{eqnarray}}
\def\beqn{\begin{eqnarray}}
\def\eeqn{\end{eqnarray}}
\def\nn{\nonumber}
\def\ba{\beq\new\begin{array}{c}}
\def\ea{\end{array}\eeq}
\def\be{\ba}
\def\ee{\ea}
\newcommand{\beas}{\begin{eqnarray*}}
\newcommand{\eeas}{\end{eqnarray*}}
\newcommand{\defi}{\stackrel{\rm def}{=}}
\newcommand{\non}{\nonumber}
\newcommand{\bquo}{\begin{quote}}
\newcommand{\enqu}{\end{quote}}
\newcommand{\m}{\tilde m}
\newcommand{\trho}{\tilde{\rho}}
\newcommand{\tn}{\tilde{n}}
\newcommand{\tN}{\tilde N}
%\newcommand{\p}{\partial}
\newcommand{\gsim}{\lower.7ex\hbox{$\;\stackrel{\textstyle>}{\sim}\;$}}
\newcommand{\lsim}{\lower.7ex\hbox{$\;\stackrel{\textstyle<}{\sim}\;$}}

%%%%%%%%%%%%%%%%%%%%%%%%%%%%%%%%%% definitions

\def\de{\partial}
\def\const{\hbox {\rm const.}}  
\def\o{\over}
\def\im{\hbox{\rm Im}}
\def\re{\hbox{\rm Re}}
\def\bra{\langle}\def\ket{\rangle}
\def\Arg{\hbox {\rm Arg}}
\def\Re{\hbox {\rm Re}}
\def\Im{\hbox {\rm Im}}
\def\diag{\hbox{\rm diag}}


%%%%%%%%%%%%%%%%%%%%%%%%%%%%%%%%%%%%%%%%%%%%%%%%%%%%%%%%%%%%%%%%%%%%

\def\QATOPD#1#2#3#4{{#3 \atopwithdelims#1#2 #4}}
\def\stackunder#1#2{\mathrel{\mathop{#2}\limits_{#1}}}
\def\stackreb#1#2{\mathrel{\mathop{#2}\limits_{#1}}}
\def\res{{\rm res}}
\def\Bf#1{\mbox{\boldmath $#1$}}
\def\balpha{{\Bf\alpha}}
\def\bbeta{{\Bf\beta}}
\def\bgamma{{\Bf\gamma}}
\def\bnu{{\Bf\nu}}
\def\bmu{{\Bf\mu}}
\def\bphi{{\Bf\phi}}
\def\bPhi{{\Bf\Phi}}
\def\bomega{{\Bf\omega}}
\def\blambda{{\Bf\lambda}}
\def\brho{{\Bf\rho}}
\def\bsigma{{\bfit\sigma}}
\def\bxi{{\Bf\xi}}
\def\bbeta{{\Bf\eta}}
\def\d{\partial}
\def\der#1#2{\frac{\d{#1}}{\d{#2}}}
\def\Im{{\rm Im}}
\def\Re{{\rm Re}}
\def\rank{{\rm rank}}
\def\diag{{\rm diag}}
\def\2{{1\over 2}}
\def\x{\stackrel{\otimes}{,}}

\def\ba{\beq\new\begin{array}{c}}
\def\ea{\end{array}\eeq}
\def\be{\ba}
\def\ee{\ea}
\def\stackreb#1#2{\mathrel{\mathop{#2}\limits_{#1}}}



\newcommand{\nfour}{${\cal N}=4\;$}
\newcommand{\none}{${\mathcal N}=1\,$}
\newcommand{\nonen}{${\mathcal N}=1$}
\newcommand{\ntwo}{${\mathcal N}=2$}
\newcommand{\ntt}{${\mathcal N}=(2,2)\,$}
\newcommand{\nzt}{${\mathcal N}=(0,2)\,$}
\newcommand{\ntwon}{${\mathcal N}=2$}
\newcommand{\ntwot}{${\mathcal N}= \left(2,2\right) $ }
\newcommand{\ntwoo}{${\mathcal N}= \left(0,2\right) $ }
\newcommand{\ntwoon}{${\mathcal N}= \left(0,2\right)$}


\newcommand{\ca}{{\mathcal A}}
\newcommand{\cell}{{\mathcal L}}
\newcommand{\cw}{{\mathcal W}}
\newcommand{\cs}{{\mathcal S}}
\newcommand{\vp}{\varphi}
\newcommand{\pt}{\partial}
\newcommand{\ve}{\varepsilon}
\newcommand{\gs}{g^{2}}
\newcommand{\zn}{$Z_N$}
\newcommand{\cd}{${\mathcal D}$}
\newcommand{\cde}{{\mathcal D}}
\newcommand{\cf}{${\mathcal F}$}
\newcommand{\cfe}{{\mathcal F}}
\newcommand{\ff}{\mc{F}}
\newcommand{\bff}{\ov{\mc{F}}}


\newcommand{\p}{\partial}
\newcommand{\wt}{\widetilde}
\newcommand{\ov}{\overline}
\newcommand{\mc}[1]{\mathcal{#1}}
\newcommand{\md}{\mathcal{D}}
\newcommand{\ml}{\mathcal{L}}
\newcommand{\mw}{\mathcal{W}}
\newcommand{\ma}{\mathcal{A}}


\newcommand{\GeV}{{\rm GeV}}
\newcommand{\eV}{{\rm eV}}
\newcommand{\Heff}{{\mathcal{H}_{\rm eff}}}
\newcommand{\Leff}{{\mathcal{L}_{\rm eff}}}
\newcommand{\el}{{\rm EM}}
\newcommand{\uflavor}{\mathbf{1}_{\rm flavor}}
\newcommand{\lgr}{\left\lgroup}
\newcommand{\rgr}{\right\rgroup}


\newcommand{\Mpl}{M_{\rm Pl}}
\newcommand{\suc}{{{\rm SU}_{\rm C}(3)}}
\newcommand{\sul}{{{\rm SU}_{\rm L}(2)}}
\newcommand{\sutw}{{\rm SU}(2)}
\newcommand{\suth}{{\rm SU}(3)}
\newcommand{\ue}{{\rm U}(1)}


\newcommand{\LN}{\Lambda_\text{SU($N$)}}
\newcommand{\sunu}{{\rm SU($N$) $\times$ U(1) }}
\newcommand{\sunun}{{\rm SU($N$) $\times$ U(1)}}
\def\cfl {$\text{SU($N$)}_{\rm C+F}$ }
\def\cfln {$\text{SU($N$)}_{\rm C+F}$}
\newcommand{\mUp}{m_{\rm U(1)}^{+}}
\newcommand{\mUm}{m_{\rm U(1)}^{-}}
\newcommand{\mNp}{m_\text{SU($N$)}^{+}}
\newcommand{\mNm}{m_\text{SU($N$)}^{-}}
\newcommand{\AU}{\mc{A}^{\rm U(1)}}
\newcommand{\AN}{\mc{A}^\text{SU($N$)}}
\newcommand{\aU}{a^{\rm U(1)}}
\newcommand{\aN}{a^\text{SU($N$)}}
\newcommand{\baU}{\ov{a}{}^{\rm U(1)}}
\newcommand{\baN}{\ov{a}{}^\text{SU($N$)}}
\newcommand{\lU}{\lambda^{\rm U(1)}}
\newcommand{\lN}{\lambda^\text{SU($N$)}}
\newcommand{\bxir}{\ov{\xi}{}_R}
\newcommand{\bxil}{\ov{\xi}{}_L}
\newcommand{\xir}{\xi_R}
\newcommand{\xil}{\xi_L}
\newcommand{\bzl}{\ov{\zeta}{}_L}
\newcommand{\bzr}{\ov{\zeta}{}_R}
\newcommand{\zr}{\zeta_R}
\newcommand{\zl}{\zeta_L}
\newcommand{\nbar}{\ov{n}}
\newcommand{\nnbar}{n\ov{n}}
\newcommand{\muU}{\mu_\text{U}}


\newcommand{\cpn}{CP$^{N-1}$\,}
\newcommand{\CPC}{CP($N-1$)$\times$C }
\newcommand{\CPCn}{CP($N-1$)$\times$C}


\newcommand{\lar}{\lambda_R}
\newcommand{\lal}{\lambda_L}
\newcommand{\larl}{\lambda_{R,L}}
\newcommand{\lalr}{\lambda_{L,R}}
\newcommand{\blar}{\ov{\lambda}{}_R}
\newcommand{\blal}{\ov{\lambda}{}_L}
\newcommand{\blarl}{\ov{\lambda}{}_{R,L}}
\newcommand{\blalr}{\ov{\lambda}{}_{L,R}}


\newcommand{\bpsi}{\ov{\psi}{}}


\newcommand{\hphi}{\hat\phi{}}
\newcommand{\hbphi}{\hat{\ov\phi}{}}
\newcommand{\hxi}{\hat\xi{}}
\newcommand{\hbxi}{\hat{\ov\xi}{}}
\newcommand{\hsigma}{\hat\sigma{}}
\newcommand{\hbsigma}{\hat{\ov\sigma}{}}
\newcommand{\hlambda}{\hat\lambda{}}
\newcommand{\hblambda}{\hat{\ov\lambda}{}}
\newcommand{\hz}{\hat z{}}
\newcommand{\hbz}{\hat{\ov z}{}}
\newcommand{\hzeta}{\hat\zeta{}}
\newcommand{\hbzeta}{\hat{\ov\zeta}{}}



\newcommand{\qt}{\wt{q}}
\newcommand{\bq}{\ov{q}}
\newcommand{\bqt}{\overline{\widetilde{q}}}


\newcommand{\eer}{\epsilon_R}
\newcommand{\eel}{\epsilon_L}
\newcommand{\eerl}{\epsilon_{R,L}}
\newcommand{\eelr}{\epsilon_{L,R}}
\newcommand{\beer}{\ov{\epsilon}{}_R}
\newcommand{\beel}{\ov{\epsilon}{}_L}
\newcommand{\beerl}{\ov{\epsilon}{}_{R,L}}
\newcommand{\beelr}{\ov{\epsilon}{}_{L,R}}


\newcommand{\bi}{{\bar \imath}}
\newcommand{\bj}{{\bar \jmath}}
\newcommand{\bk}{{\bar k}}
\newcommand{\bl}{{\bar l}}
\newcommand{\bmm}{{\bar m}}


\newcommand{\nz}{{n^{(0)}}}
\newcommand{\no}{{n^{(1)}}}
\newcommand{\bnz}{{\ov{n}{}^{(0)}}}
\newcommand{\bno}{{\ov{n}{}^{(1)}}}
\newcommand{\Dz}{{D^{(0)}}}
\newcommand{\Do}{{D^{(1)}}}
\newcommand{\bDz}{{\ov{D}{}^{(0)}}}
\newcommand{\bDo}{{\ov{D}{}^{(1)}}}
\newcommand{\sigz}{{\sigma^{(0)}}}
\newcommand{\sigo}{{\sigma^{(1)}}}
\newcommand{\bsigz}{{\ov{\sigma}{}^{(0)}}}
\newcommand{\bsigo}{{\ov{\sigma}{}^{(1)}}}


\newcommand{\rrenz}{{r_\text{ren}^{(0)}}}
\newcommand{\bren}{{\beta_\text{ren}}}


\newcommand{\Tr}{\text{Tr}}
\newcommand{\Ts}{\text{Ts}}
\newcommand{\dm}{\hat{{\scriptstyle \Delta} m}}
\newcommand{\dmdag}{\hat{{\scriptstyle \Delta} m}{}^\dag}
\newcommand{\mhat}{\widehat{m}}
\newcommand{\deltam}{{\scriptstyle \Delta} m}
\newcommand{\nvac}{\vec{n}{}_\text{vac}}


\newcommand{\ie}{{\it i.e.}~}
\newcommand{\eg}{{\it e.g.}~}
\newcommand{\ansatz}{{\it ansatz} }


\begin{document}

%%%%%%%%%%%%%%%%%%%%%%%%%%%%%%%%%%%%%%%%%%%%%%%%%%%%%%%%%%%%%%%%%%%%%%%%%%%%%%%%
%                                                                              %
%                                                                              %
%                            T I T L E   P A G E                               %
%                                                                              %
%                                                                              %
%%%%%%%%%%%%%%%%%%%%%%%%%%%%%%%%%%%%%%%%%%%%%%%%%%%%%%%%%%%%%%%%%%%%%%%%%%%%%%%%
\begin{titlepage}


\begin{flushright}
FTPI-MINN-XX/XX, UMN-TH-XXXX/XX\\
August 18/2014/DRAFT
\end{flushright}

\vspace{1.0cm}

\begin{center}
{  \Large \bf  Appendix only}
\end{center}


\end{titlepage}

%%%%%%%%%%%%%%%%%%%%%%%%%%%%%%%%%%%%%%%%%%%%%%%%%%%%%%%%%%%%%%%%%%%%%%%%%%%%%%%%
%                                                                              %
%                                                                              %
%                              A P P E N D I C E S                             %
%                                                                              %
%                                                                              %
%%%%%%%%%%%%%%%%%%%%%%%%%%%%%%%%%%%%%%%%%%%%%%%%%%%%%%%%%%%%%%%%%%%%%%%%%%%%%%%%
\newpage
\appendix
\setcounter{equation}{0}


%%%%%%%%%%%%%%%%%%%%%%%%%%%%%%%%%%%%%%%%%%%%%%%%%%%%%%%%%%%%%%%%%%%%%%%%%%%%%%%%
%                                                                              %
%                               N O T A T I O N S                              %
%                                                                              %
%%%%%%%%%%%%%%%%%%%%%%%%%%%%%%%%%%%%%%%%%%%%%%%%%%%%%%%%%%%%%%%%%%%%%%%%%%%%%%%%
\section{Notations}
\label{app:notations}


%%%%%%%%%%%%%%%%%%%%%%%%%%%%%%%%%%%%%%%%%%%%%%%%%%%%%%%%%%%%%%%%%%%%%%%%%%%%%%%%
%                                                                              %
%                     T W I S T E D   S U P E R F I E L D S                    %
%                                                                              %
%%%%%%%%%%%%%%%%%%%%%%%%%%%%%%%%%%%%%%%%%%%%%%%%%%%%%%%%%%%%%%%%%%%%%%%%%%%%%%%%
\subsection{Twisted superfields}


	Twisted superfields exist in two dimensions and are defined by a ``twisted'' chirality condition
\beq
\label{gentwist}
	D_L\, \Sigma    ~~=~~    \ov D{}_R\, \Sigma    ~~=~~    0\,.
\eeq
	Analogously, for twisted anti-chiral superfields,
\beq
\label{genantitwist}
	D_R\, \ov\Sigma    ~~=~~    \ov D{}_L\, \ov\Sigma    ~~=~~    0\,.
\eeq
	The reason these conditions look so similar to those for chiral superfields is that
	there is no ideomatic difference between chiral and twisted-chiral superfields.
	To say more, they are interchanged by the action of mirror symmetry ---
	the transposition of supercharges turns condition \eqref{gentwist} into 
	the one for a chiral superfield.
	This way, as in the case with the chiral superfields, 
	the constraints \eqref{gentwist} and \eqref{genantitwist} are solved by letting
	the superfields be arbitrary functions of ``chiral'' variables $ \wt y^\mu $,
\begin{align*}
%%
	& \Sigma    ~~=~~    \Sigma(\wt y{}^\mu)
	&
	& \ov\Sigma    ~~=~~    \ov\Sigma(\ov{\wt y}{}^\mu)
	\\[2mm]
%%
	\wt y{}^0    ~~=~~ &    x^0  ~+~ i \left( \ov\theta{}_R\theta_R  ~-~  \ov\theta{}_L\theta_L \right)
	&
	\ov{\wt y}{}^0     ~~=~~ &    x^0  ~-~ i \left( \ov\theta{}_R\theta_R  ~-~  \ov\theta{}_L\theta_L \right)
	\\[2mm]
%%
	\wt y{}^3    ~~=~~ &    x^3  ~+~ \left( \ov\theta{}_R\theta_R  ~+~  \ov\theta{}_L\theta_L \right)
	&
	\ov{\wt y}{}^3    ~~=~~ &    x^3  ~-~ \left( \ov\theta{}_R\theta_R  ~+~  \ov\theta{}_L\theta_L \right)
	\,,
\end{align*}
	after which they will have the usual ``chiral'' component expansion
\begin{align*}
%%
	\Sigma(\wt y)    & ~~=~~    \sigma(\wt y)  ~~-~~  \sqrt{2}\, \theta_R \ov\lambda{}_L
						   ~~+~~  \sqrt{2}\, \ov\theta{}_L \lambda_R
						   ~~+~~  \sqrt{2}\, \theta_R \ov\theta{}_L\, \wt{F}
	\\[2mm]
%%
	\ov\Sigma(\ov{\wt y})    & ~~=~~    \ov\sigma(\ov{\wt y})  ~~-~~ \sqrt{2}\, \theta_L \ov\lambda{}_R
								   ~~+~~ \sqrt{2}\, \ov\theta{}_R \lambda_L
								   ~~+~~ \sqrt{2}\, \theta_L \ov\theta{}_R\, \ov{\wt F}
	\,.
\end{align*}
	Here we understand that each function on the right hand side depends on 
	$ \wt y{}^\mu $ and $ \ov{\wt y}{}^\mu $, correspondingly.

	Exactly the same way as with chiral superfields, one constructs {\it twisted superpotentials}
	$ \wt\cw(\Sigma) $, just as functions that depend on $ \Sigma $ holomorphically.
	One then performs the twisted $ d^2\tilde\theta $ integration as,
\beq
	\int\, d^2\tilde\theta\, \wt{\cw}(\Sigma)    ~~=~~    \frac{1}{2}\,\ov D{}_L\, D_R\, \wt{\cw}(\Sigma)\Big|\,,
	\qquad
	\int\, d^2\ov{\tilde\theta}\, \ov{\wt{\cw}}(\ov\Sigma)    ~~=~~    \frac{1}{2}\,\ov D{}_R\, D_L\, \ov{\wt{\cw}}(\ov\Sigma)\Big|\,.
\eeq
	And, of course, one can perform the full superspace integration of twisted superfields, provided
	that this holomorphicity is broken ({\it e.g.} by putting both chiral and anti-chiral factors),
\beq
	\int\, d^4\theta\, \ov\Sigma\, \Sigma\,,
\eeq
	or the result will obviously be a total derivative.


	One famous example of a twisted superfield is the {\it fieldstrength}
	of a \ntwot gauge supermultiplet $ V $,
\begin{align}
%%
	\Sigma    & ~~=~~    \frac{i}{\sqrt 2}\, D_L\, \ov D{}_R\, V\,,
	&
	\ov \Sigma    & ~~=~~    \frac{i}{\sqrt 2}\, D_R\, \ov D{}_L\, V\,.
\end{align}
	In components it takes the form
\begin{align*}
%%
	\Sigma(\wt y)    & ~~=~~    \sigma(\wt y)  ~~-~~  \sqrt{2}\, \theta_R \ov\lambda{}_L
						   ~~+~~  \sqrt{2}\, \ov\theta{}_L \lambda_R
						   ~~+~~  \sqrt{2}\, \theta_R \ov\theta{}_L \lgr D ~-~ i\, F_{03} \rgr
	\\[2mm]
%%
	\ov\Sigma(\ov{\wt y})    & ~~=~~    \ov\sigma(\ov{\wt y})  ~~-~~ \sqrt{2}\, \theta_L \ov\lambda{}_R
								   ~~+~~ \sqrt{2}\, \ov\theta{}_R \lambda_L
								   ~~+~~ \sqrt{2}\, \theta_L \ov\theta{}_R \lgr D ~+~ i\, F_{03} \rgr
	.
\end{align*}




%%%%%%%%%%%%%%%%%%%%%%%%%%%%%%%%%%%%%%%%%%%%%%%%%%%%%%%%%%%%%%%%%%%%%%%%%%%%%%%%
%                                                                              %
%                     N = ( 0 , 2 )   S U P E R F I E L D S                    %
%                                                                              %
%%%%%%%%%%%%%%%%%%%%%%%%%%%%%%%%%%%%%%%%%%%%%%%%%%%%%%%%%%%%%%%%%%%%%%%%%%%%%%%%
\subsection{\boldmath{\ntwoon} superfields}

	We define \ntwoo superspace via reduction of \ntwot superspace by putting
\beq
	\theta_L    ~~=~~    \ov\theta{}_L    ~~=~~    0\,.
\eeq
	Each chiral and twisted-chiral \ntwot superfield this way splits into two \ntwoo superfields.
	They still, however, retain their property of holomorphicity.
	While chiral superfields are usually described as being dependent on
	a ``holomorphic'' variable $ y^\mu $,
\beq
	y^\mu    ~~=~~    x^\mu    ~~+~~    i\,\ov{\theta\sigma}{}_\mu\theta\,,
\eeq
	and twisted-chiral as dependent on a twisted variable $ \wt y{}^\mu $,
\beq
	\wt y{}^\mu    ~~=~~    x^\mu    ~~+~~    i\,\ov{\theta\wt\sigma}{}_\mu\theta\,,
\eeq
	the distinction between the two variables vanishes upon reduction to \ntwoo superspace.
	As a result, \ntwoo superspace defines only one kind of holomorphic superfields --- 
	chiral \ntwoo superfields, which depend upon the reduced variables $ \upsilon^\mu $
\begin{align}
%
\notag
	\upsilon^0 &    ~~=~~    x^0  ~~+~~  i\,\ov\theta{}_R\theta_R
	\\[2mm]
%
	\upsilon^3 &    ~~=~~    x^3  ~~+~~  i\,\ov\theta{}_R\theta_R\,.
\end{align}

	To make a distinction with \ntwot superfields, we denote the \ntwoo superfields
	by symbols with a hat on top --- $ \hsigma $, $ \hxi $, {\it etc}.

	Chiral superfields $ \Phi(y) $ split into \ntwoo superfields 
	$ \hphi(\upsilon) $ and $ \hxi(\upsilon) $,
\begin{align}
%
\notag
	\Phi(y) &    ~~\longrightarrow~~    \hphi(\upsilon)  ~~+~~  \sqrt{2}\,\theta_L\,\hxi(\upsilon)\,,
	\\[2mm]
%
\label{Phisplit}
	\ov\Phi(\ov y) &    ~~\longrightarrow~~    \hbphi(\ov\upsilon)  ~~-~~  \sqrt{2}\,\ov\theta{}_L\,\hbxi(\ov\upsilon)\,,
\end{align}
	while twisted-chiral superfields $ \Sigma(\wt y) $ split into 
	$ \hsigma(\upsilon) $ and $ \hlambda(\upsilon) $,
\begin{align}
%
\notag
	\Sigma(\wt y) &    ~~\longrightarrow~~    \hsigma(\upsilon)  ~~+~~  \sqrt{2}\,\ov\theta{}_L\,\hlambda(\upsilon)\,,
	\\[2mm]
%
\label{Sigmasplit}
	\ov\Sigma(\ov{\wt y}) &    ~~\longrightarrow~~    \hbsigma(\ov\upsilon)  ~~-~~  \sqrt{2}\,\theta_L\,\hblambda(\upsilon)\,.
\end{align}
	We alert that relations \eqref{Phisplit} and \eqref{Sigmasplit} have only symbolical meaning demonstrating
	the effect of splitting, while there is no equality:
	the right-hand sides have incomplete dependence on $ \theta_L $, $ \ov\theta{}_L $.

	The individual \ntwoo superfields have a quite straightforward component expansion,
\begin{align}
%
\notag
	\hphi(\upsilon) &    ~~=~~    \phi  ~~-~~  \sqrt{2}\,\theta_R\,\psi_L\,,
	&
	\hxi(\upsilon) &    ~~=~~    \psi_R  ~~+~~  \sqrt{2}\,\theta_R\,F\,,
	\\[2mm]
%
	\hbphi(\ov\upsilon) &    ~~=~~    \ov \phi  ~~+~~  \sqrt{2}\,\ov{\theta_L \psi}{}_L\,,
	&
	\hbxi(\ov\upsilon) &    ~~=~~    \ov\psi{}_R  ~~+~~  \sqrt{2}\,\ov{\theta_R F}\,,
\end{align}
	and similarly do the ones that arise from splitting of the the twisted chiral superfield $ \Sigma(\wt y) $, 
\begin{align}
%
\notag
	\hsigma(\upsilon) &    ~~=~~    \sigma  ~~-~~  \sqrt{2}\,\theta_R\,\ov\lambda{}_L\,,
	&
	\hlambda(\upsilon) &    ~~=~~    \lambda_R  ~~-~~  \theta_R\,\wt F\,,
	\\[2mm]
%
	\hbsigma(\ov\upsilon) &    ~~=~~    \ov\sigma  ~~+~~  \sqrt{2}\,\ov\theta{}_R\,\lambda_L\,,
	&
	\hblambda(\ov\upsilon) &    ~~=~~    \ov\lambda{}_R  ~~-~~  \ov\theta{}_R\,\ov{\wt F}\,.
\end{align}

	
	It is interesting to note that the simple structure of splitting shown in \eqref{Phisplit} and \eqref{Sigmasplit}
	makes fermionic superfields in \ntwoo superspace much more ubiqutious than in \ntwot superspace.
	And the first example to this is the \ntwoo superpotential.
	It has to be fermionic because the integration over {\it half} of the \ntwoo superspace $ d\theta_R $ is such.
	A superpotential can be constructed using an arbitrary holomorphic function --- say $ J(\hsigma) $, and
	by multiplying it by an arbitrary (but still chiral) fermionic multiplet --- say $ \hat{\rho} $,
\beq
	\int\, d\theta_R\, \hat{\rho}\, J(\hsigma)\,.
\eeq

	``Full'' superspace integrals can conventionally be built using both chiral and antichiral fields,
\beq
	\int\, d^2\theta_R\,\, \hxi\, \hbxi    ~~=~~    \ov\psi{}_R\,i\p_L \psi_R  ~~+~~  \ov F\,F\,.
\eeq



%%%%%%%%%%%%%%%%%%%%%%%%%%%%%%%%%%%%%%%%%%%%%%%%%%%%%%%%%%%%%%%%%%%%%%%%%%%%%%%%
%                                                                              %
%                                                                              %
%           E X P A N S I O N   O F   E F F E C T I V E   A C T I O N          %
%                                                                              %
%                                                                              %
%%%%%%%%%%%%%%%%%%%%%%%%%%%%%%%%%%%%%%%%%%%%%%%%%%%%%%%%%%%%%%%%%%%%%%%%%%%%%%%%
\section{Component Expansion of the Effective Action}
\label{app:expansion}

	Here we give the complete component expansion of expression \eqref{Lsuper}. 
	Typically, one would be interested in a specific limit of this expression,
	such as the bosonic part of it, or the constant bosonic part, or 
	an approximation in the certain number of space-time derivatives
	(some approximations, however, are easier to derive from 
	the series representation \eqref{sseries}).

	We make a remark that, according to \cite{1p}, the whole fermionic part of
	the below effective Lagrangian can be obtained from its bosonic part simply
	via a replacement
\beq
	\sqrt{2}\sigma    ~~\longrightarrow~~    \sqrt{2}\sigma  ~~+~~  2\, \frac{i\,\ov\lambda{}_L\lambda_R}{iD + F_{03}}\,.
\eeq

	It is useful to know the lowest component of the superfield $ S $,
\begin{align*}
%
	s    ~~=~~
	S \Big| &    ~~=~~    \frac{ \sqrt{2}\ov\sigma\, ( iD \,-\, F_{03} ) ~-~ 2\,i\, \ov\lambda{}_R \lambda_L }
						{ \big( \sqrt{2}\, \ov\sigma \big)^2 }\,,
	\\[2mm]
%
	\ov s    ~~=~~
	\ov S \Big| &    ~~=~~    \frac{ \sqrt{2}\sigma\, ( iD \,+\, F_{03} ) ~-~ 2\,i\, \ov\lambda{}_L \lambda_R }
						{ \big( \sqrt{2}\, \sigma \big)^2 }\,.
\end{align*}
	In the expression below, however, we extensively make use of the lowest component of the ratio $ S / (\sqrt{2}\Sigma) $,
	which we denote as $ p $,
\begin{align*}
%
	p    ~~=~~
	\frac{S}{\sqrt{2}\Sigma} \bigg| &    ~~=~~
		\frac{1}{\big|\sqrt{2}\sigma\big|^2}
		\lgr
			iD \,-\, F_{03}
			~~-~~
			\frac{2\, i\sqrt{2}\sigma \ov\lambda{}_R\lambda_L}{\big|\sqrt{2}\sigma\big|^2}
		\rgr,
	\\[2mm]
%
	\ov p    ~~=~~
	\frac{\ov S}{\sqrt{2}\ov\Sigma} \bigg| &    ~~=~~
		\frac{1}{\big|\sqrt{2}\sigma\big|^2}
		\lgr
			iD \,+\, F_{03}
			~~-~~
			\frac{2\, i\sqrt{2}\ov{\sigma\lambda}{}_L\lambda_R}{\big|\sqrt{2}\sigma\big|^2}
		\rgr.
\end{align*}

	
	We have,
\begingroup
\allowdisplaybreaks
\begin{align}
%
\notag
	& \frac{4\pi}{N}\, \cell    ~~=~~     
			iD  ~~-~~  F_{03}\, \log\, \frac{\sqrt{2}\sigma}{\sqrt{2}\ov\sigma}
		~~-~~ i\, \frac{ \sqrt{2}\sigma\ov\lambda{}_R\lambda_L ~+~ \sqrt{2}\ov{\sigma\lambda}{}_L\lambda_R }
				{ \big|\sqrt{2}\sigma\big|^2 }
		~~-
	\\[2mm]
%
\notag
	&
		~~-~~  \frac{1}{4}\, \ln \sqrt{2}\sigma\, \Box\, \ln \sqrt{2}\sigma
		~~-~~  \frac{1}{4}\, \ln \sqrt{2}\ov\sigma\, \Box\, \ln \sqrt{2}\ov\sigma
		~~-
	\\[2mm]
%
\notag
	&
		~~-~~  \frac{1}{2} \Big\lgroup iD  \,+\,  F_{03}  \,+\,  \frac{1}{2}\,\Box\,\ln \sqrt{2}\ov\sigma \Big\rgroup
			\ln \lgr \big|\sqrt{2}\sigma\big|^2  \,+\,  iD  \,-\,  F_{03}  
			\,-\,  2i\, \frac{\ov\lambda{}_R\lambda_L}{\sqrt{2}\ov\sigma} \rgr 
		~-
	\\[2mm]
%
\notag
	&
		~~-~~  \frac{1}{2} \Big\lgroup iD  \,-\,  F_{03}  \,+\,  \frac{1}{2}\,\Box\,\ln \sqrt{2}\sigma \Big\rgroup
			\ln \lgr \big|\sqrt{2}\sigma\big|^2  \,+\,  iD  \,+\,  F_{03}
			\,-\,  2i\, \frac{\ov\lambda{}_L\lambda_R}{\sqrt{2}\sigma} \rgr
	\\[2mm]
%
\notag
	&
		~~+~~  
			\frac{ 
				-\, \ov\lambda{}_R\, i\overleftarrow{\md}_L \lambda_R\,   \,+\, \ov{\lambda}{}_L\, i\md_R \lambda_L \,-\,
				\frac{\displaystyle 1}{\displaystyle 2}\, i\,\sqrt{2}\ov{\sigma \lambda}{}_L \lambda_R
				\,+\, \frac{\displaystyle 1}{\displaystyle 2}\,
					\frac{\displaystyle 1}{\displaystyle\sqrt{2}\ov\sigma}\, 
					i\, \ov\lambda{}_R \overleftarrow{\md}{}_L\, \md_R \lambda_L
			}
			{
				\big|\sqrt{2}\sigma\big|^2 \,+\, iD \,-\, F_{03} 
				\,-\, 2i\,\frac{\displaystyle \ov\lambda{}_R\lambda_L}{\displaystyle \sqrt{2}\ov\sigma}
			}
	\\[2mm]
%
\notag
	&
		~~+~~  
			\frac{
				\ov\lambda{}_R\, i\md_L\lambda_R \,-\, \ov\lambda{}_L\, i\overleftarrow{\md}_R \lambda_L \,-\,
				\frac{\displaystyle 1}{\displaystyle 2}\, i\,\sqrt{2}\sigma \ov\lambda{}_R \lambda_L
				\,+\, \frac{\displaystyle 1}{\displaystyle 2}\,
					\frac{\displaystyle 1}{\displaystyle\sqrt{2}\sigma}\,
					i\, \ov\lambda{}_L \overleftarrow{\md}_R\, \md_L\lambda_R
			}
			{
				\big|\sqrt{2}\sigma\big|^2 \,+\, iD \,+\, F_{03} 
				\,-\, 2i\,\frac{\displaystyle \ov\lambda{}_L\lambda_R}{\displaystyle \sqrt{2}\sigma}
			}
	\\[2mm]
%
\notag
	&
		~~-~~  \frac{1}{4}\,
			\frac{
				\big|\sqrt{2}\sigma\big|^2
				\lgr iD \,+\, F_{03} \,+\, \Box\, \ln \sqrt{2}\ov\sigma \rgr
			}
			{
				\big|\sqrt{2}\sigma\big|^2 \,+\, iD \,-\, F_{03} 
				\,-\, 2i\,\frac{\displaystyle \ov\lambda{}_R\lambda_L}{\displaystyle \sqrt{2}\ov\sigma}
			}
		~~-~~  \frac{1}{4}\,
			\frac{
				\big|\sqrt{2}\sigma\big|^2
				\lgr iD \,-\, F_{03} \,+\, \Box\, \ln \sqrt{2}\sigma \rgr
			}
			{
				\big|\sqrt{2}\sigma\big|^2 \,+\, iD \,+\, F_{03} 
				\,-\, 2i\,\frac{\displaystyle \ov\lambda{}_L\lambda_R}{\displaystyle \sqrt{2}\sigma}
			}
	\\[2mm]
%
\notag
	&
		~~-~~  \frac{1}{2}\, \big|\sqrt{2}\sigma\big|^2
			\lgr 1 \,+\, \frac{1}{p} \rgr \!
			\frac{ 
				-\, \ov\lambda{}_R\, i\overleftarrow{\md}_L \lambda_R  \,+\, \ov{\lambda}{}_L\, i\md_R \lambda_L \,+\,
				i\,\sqrt{2}\ov{\sigma \lambda}{}_L \lambda_R
				\,+\, \frac{\displaystyle 1}{\displaystyle \sqrt{2}\ov\sigma}\,
					i\, \ov\lambda{}_R \overleftarrow{\md}_L\, \md_R\lambda_L
			}
			{
			\lgr  
				\big|\sqrt{2}\sigma\big|^2 \,+\, iD \,-\, F_{03} 
				\,-\, 2i\,\frac{\displaystyle \ov\lambda{}_R\lambda_L}{\displaystyle \sqrt{2}\ov\sigma}  
			\rgr^2
			}
	\\[2mm]
%
\notag
	&
		~~-~~  \frac{1}{2}\, \big|\sqrt{2}\sigma\big|^2
			\lgr 1 \,+\, \frac{1}{\ov p} \rgr\!
			\frac{
				\ov\lambda{}_R\, i\md_L\lambda_R \,-\, \ov\lambda{}_L\, i\overleftarrow{\md}_R \lambda_L \,+\,
				i\,\sqrt{2}\sigma \ov\lambda{}_R \lambda_L
				\,+\, \frac{\displaystyle 1}{\displaystyle \sqrt{2}\sigma}\,
					i\, \ov\lambda{}_L \overleftarrow{\md}_R\, \md_L\lambda_R
			}
			{
			\lgr
				\big|\sqrt{2}\sigma\big|^2 \,+\, iD \,+\, F_{03} 
				\,-\, 2i\,\frac{\displaystyle \ov\lambda{}_L\lambda_R}{\displaystyle \sqrt{2}\sigma}
			\rgr^2
			}
	\\[2mm]
%
	&
		~~+~~  \frac{1}{2\, p^3}
			\Bigg\lgroup
				-\, p^2\, \big( iD \,+\ F_{03} \big)
				~+~  \frac{1}{2}\, p\, \Box \ln \sqrt{2}\ov\sigma
				~-~  2\, p^2\, i\, \frac{\ov\lambda{}_L \lambda_R}{\sqrt{2}\sigma}  ~+~
	\\[2mm]
%
\notag
	&
		\phantom{~~+~~  \frac{1}{2\, p^3} \Bigg\lgroup}
				+\, 2\, i\, \frac{1}{\sqrt{2}\sigma}\, 
					\frac{\ov\lambda{}_R \overleftarrow{\md}_L\, \md_R\lambda_L}{\big(\sqrt{2}\ov\sigma\big)^2}
				~-~  2\, p\, 
					\frac{\ov\lambda{}_L\, i\md_R \lambda_L ~-~ \ov\lambda{}_R\, i\overleftarrow{\md}_L \lambda_R}{\big|\sqrt{2}\sigma\big|^2}
			\Bigg\rgroup\!\!
		\cdot \ln \big( 1 ~+~ p \big)
	\\[2mm]
%
\notag
	&
		~~+~~  \frac{1}{2\, \ov p{}^3}
			\Bigg\lgroup
				-\, \ov p{}^2\, \big( iD \,-\, F_{03} \big)
				~+~  \frac{1}{2}\, \ov p\, \Box \ln \sqrt{2}\sigma
				~-~  2\, \ov p{}^2\, i\, \frac{\ov\lambda{}_R \lambda_L}{\sqrt{2}\ov\sigma}  ~+~
	\\[2mm]
%
\notag
	&
		\phantom{~~+~~  \frac{1}{2\, \ov p{}^3} \Bigg\lgroup}
				+\, 2\, i\, \frac{1}{\sqrt{2}\ov\sigma}\,
					\frac{\ov\lambda{}_L \overleftarrow{\md}_R\, \md_L\lambda_L}{\big(\sqrt{2}\sigma\big)^2}
				~-~  2\, \ov p\,
					\frac{\ov\lambda{}_R\, i\md_L \lambda_R ~-~ \ov\lambda{}_L\, i\overleftarrow{\md}_R \lambda_L}{\big|\sqrt{2}\sigma\big|^2}
			\Bigg\rgroup\!\!
		\cdot \ln \big( 1 ~+~ \ov p \big)
	\\[2mm]
%
\notag
	&
		~~+~~  \frac{1}{2\, p^2}\, i
		\lgr
			\frac{
				-\, \sqrt{2}\ov{\sigma\lambda}{}_L 
				\,-\, \ov\lambda{}_R \overleftarrow{\md}_L
			}
			{
				\big|\sqrt{2}\sigma\big|^2 \,+\, iD \,-\, F_{03} 
				\,-\, 2i\,\frac{\displaystyle \ov\lambda{}_R\lambda_L}{\displaystyle \sqrt{2}\ov\sigma}
			}
			~+~
			\frac{\ov\lambda{}_L}{\sqrt{2}\sigma}
		\rgr
		\!\!
		\lgr
			2\,p\,\lambda_R
			~+~
			\frac{\md_R \lambda_L}{\sqrt{2}\ov\sigma}
		\rgr
	\\[2mm]
%
\notag
	&
		~~+~~  \frac{1}{2\, \ov p{}^2}\, i
		\lgr
			\frac{
				-\, \sqrt{2}\sigma\ov\lambda{}_R
				\,-\, \ov\lambda{}_L \overleftarrow{\md}_R
			}
			{
				\big|\sqrt{2}\sigma\big|^2 \,+\, iD \,+\, F_{03} 
				\,-\, 2i\,\frac{\displaystyle \ov\lambda{}_L\lambda_R}{\displaystyle \sqrt{2}\sigma}
			}
			~+~
			\frac{\ov\lambda{}_R}{\sqrt{2}\ov\sigma}
		\rgr
		\!\!
		\lgr
			2\, \ov p\, \lambda_L
			~+~
			\frac{\md_L \lambda_R}{\sqrt{2}\sigma}
		\rgr
	\\[2mm]
%
\notag
	&
		~~+~~  \frac{1}{2\, p^2}\, i
		\lgr
			-\, 2\, p\, \ov\lambda{}_L
			~+~
			\frac{\ov\lambda{}_R \overleftarrow{\md}_L}{\sqrt{2}\ov\sigma}
		\rgr
		\!\!
		\lgr
			\frac{
				\sqrt{2}\ov\sigma\lambda_R
				\,-\, \md_R \lambda_L
			}
			{
				\big|\sqrt{2}\sigma\big|^2 \,+\, iD \,-\, F_{03} 
				\,-\, 2i\,\frac{\displaystyle \ov\lambda{}_R\lambda_L}{\displaystyle \sqrt{2}\ov\sigma}
			}
			~-~
			\frac{\lambda_R}{\sqrt{2}\sigma}
		\rgr
	\\[2mm]
%
\notag
	&
		~~+~~  \frac{1}{2\, \ov p{}^2}\, i
		\lgr
			-\, 2\, \ov p\, \ov\lambda{}_R
			~+~
			\frac{\ov\lambda{}_L \overleftarrow{\md}_R}{\sqrt{2}\sigma}
		\rgr
		\!\!
		\lgr
			\frac{
				\sqrt{2}\sigma\lambda_L
				\,-\, \md_L\lambda_R
			}
			{
				\big|\sqrt{2}\sigma\big|^2 \,+\, iD \,+\, F_{03} 
				\,-\, 2i\,\frac{\displaystyle \ov\lambda{}_L\lambda_R}{\displaystyle \sqrt{2}\sigma}
			}
			~-~
			\frac{\lambda_L}{\sqrt{2}\ov\sigma}
		\rgr
	\\[2mm]
%
\notag
	&
		~~-~~  \frac{1}{4\, p}\, \big|\sqrt{2}\sigma\big|^2\,
			\frac{iD \,+\, F_{03} \,+\, \Box \ln \sqrt{2}\ov\sigma}
			{
				\big|\sqrt{2}\sigma\big|^2 \,+\, iD \,-\, F_{03} 
				\,-\, 2i\,\frac{\displaystyle \ov\lambda{}_R\lambda_L}{\displaystyle \sqrt{2}\ov\sigma}
			}
		~~+~~  \frac{1}{4}\, \big|\sqrt{2}\sigma\big|^2\, \frac{\ov p}{p}
	\\[2mm]
%
\notag
	&
		~~-~~  \frac{1}{4\, \ov p}\, \big|\sqrt{2}\sigma\big|^2\,
			\frac{iD \,-\, F_{03} \,+\, \Box \ln \sqrt{2}\sigma}
			{
				\big|\sqrt{2}\sigma\big|^2 \,+\, iD \,+\, F_{03} 
				\,-\, 2i\,\frac{\displaystyle \ov\lambda{}_L\lambda_R}{\displaystyle \sqrt{2}\sigma}
			}
		~~+~~  \frac{1}{4}\, \big|\sqrt{2}\sigma\big|^2\, \frac{p}{\ov p}
	\,.
\end{align}
\endgroup
\pagebreak                          %% May remove 'pagebreak' in the final version


%%%%%%%%%%%%%%%%%%%%%%%%%%%%%%%%%%%%%%%%%%%%%%%%%%%%%%%%%%%%%%%%%%%%%%%%%%%%%%%%
%                                                                              %
%                                                                              %
%                            B I B L I O G R A P H Y                           %
%                                                                              %
%                                                                              %
%%%%%%%%%%%%%%%%%%%%%%%%%%%%%%%%%%%%%%%%%%%%%%%%%%%%%%%%%%%%%%%%%%%%%%%%%%%%%%%%
\small
\begin{thebibliography}{99}

  \bibitem{0}
E.~Witten,
 {\em Instantons, the Quark Model, and the 1/n Expansion,}
  Nucl.\ Phys.\ B {\bf 149}, 285 (1979).
  %%CITATION = NUPHA,B149,285;%%
  %678 citations counted in INSPIRE as of 15 Aug 2014
  
   \bibitem{1p}
   A.~D'Adda, P.~Di Vecchia and M.~Luscher,
{\em Confinement and Chiral Symmetry Breaking in CP(N-1) Models with Quarks,}
  Nucl.\ Phys.\ B {\bf 152}, 125 (1979).
  %%CITATION = NUPHA,B152,125;%%
  %354 citations counted in INSPIRE as of 18 Aug 2014
  
    \bibitem{SYhet}
    M.~Shifman and A.~Yung,
{\em Large-N Solution of the Heterotic N=(0,2) Two-Dimensional CP(N-1) Model,}
  Phys.\ Rev.\  D {\bf 77}, 125017 (2008)
  [arXiv:0803.0698 [hep-th]].
  %%CITATION = PHRVA,D77,125017;%%

  \bibitem{1}
  A.~D'Adda, A.~C.~Davis, P.~Di Vecchia and P.~Salomonson,
 {\em An Effective Action for the Supersymmetric {CP}${(N-1)}$ Model,}
  Nucl.\ Phys.\ B {\bf 222}, 45 (1983).
  %%CITATION = NUPHA,B222,45;%%
  %63 citations counted in INSPIRE as of 15 Aug 2014
  
    \bibitem{2}
    S.~Cecotti and C.~Vafa,
  {\em On classification of ${\mathcal N}=2$ supersymmetric theories,}
  Commun.\ Math.\ Phys.\  {\bf 158}, 569 (1993)
  [hep-th/9211097].
  %%CITATION = HEP-TH/9211097;%%
  %206 citations counted in INSPIRE as of 15 Aug 2014
  
      \bibitem{3}
      E.~Witten,
{\em Phases of ${\mathcal N}=2$ theories in two-dimensions,}
  Nucl.\ Phys.\ B {\bf 403}, 159 (1993)
  [hep-th/9301042].
  %%CITATION = HEP-TH/9301042;%%
  %898 citations counted in INSPIRE as of 15 Aug 2014
  
%%  
  
  \bibitem{Veneziano}
  G.~Veneziano and S.~Yankielowicz,
  {\em An Effective Lagrangian for the Pure ${\mathcal N}=1$ Supersymmetric Yang-Mills Theory,}
  Phys.\ Lett.\ B {\bf 113} (1982) 231.
  %%CITATION = PHLTA,B113,231;%%
  %645 citations counted in INSPIRE as of 18 Aug 2014
  
   \bibitem{BSYhet}
  P.~A.~Bolokhov, M.~Shifman and A.~Yung,
  {\em Large-N Solution of the Heterotic CP(N-1) Model with Twisted Masses,}
  Phys.\ Rev.\ D {\bf 82}, 025011 (2010)
  [arXiv:1001.1757 [hep-th]].
  %%CITATION = ARXIV:1001.1757;%%
  %9 citations counted in INSPIRE as of 18 Aug 2014

  \bibitem{EdTo}
   M.~Edalati and D.~Tong,
 {\em Heterotic vortex strings,}
  JHEP {\bf 0705}, 005 (2007)
  [arXiv:hep-th/0703045].
  %%CITATION = JHEPA,0705,005;%%
  
  \bibitem{SY1}
  M.~Shifman and A.~Yung,
  {\em Heterotic Flux Tubes in ${\mathcal N}=2$ SQCD with ${\mathcal N}=1$ Preserving Deformations,}
  Phys.\ Rev.\  D {\bf 77}, 125016 (2008)
  [arXiv:0803.0158 [hep-th]].
  %%CITATION = PHRVA,D77,125016;%%

  \bibitem{BSY1}
  P.~A.~Bolokhov, M.~Shifman and A.~Yung,
  {\em Description of the Heterotic String Solutions in U(N) SQCD,}
  Phys. \ Rev. \ D {\bf 79}, 085015 (2009) (Erratum: Phys. Rev. D 80, 049902 (2009))
  [arXiv:0901.4603 [hep-th]].
  %%CITATION = ARXIV:0901.4603;%%
%%  
%%  \bibitem{BSY2}
%%  P.~A.~Bolokhov, M.~Shifman and A.~Yung,
%%  %``Description of the Heterotic String Solutions in the M Model,''
%%  Phys. \ Rev. \ D {\bf 79}, 106001 (2009) (Erratum: Phys. Rev. D 80, 049903 (2009))
%%  [arXiv:0903.1089 [hep-th]].
%%  %%CITATION = ARXIV:0903.1089;%%  
%%  

  \bibitem{BSY3}
  P.~A.~Bolokhov, M.~Shifman and A.~Yung,
  %``Heterotic N=(0,2) CP(N-1) Model with Twisted Masses,''
  Phys.\ Rev.\  D {\bf 81}, 065025 (2010)
  [arXiv:0907.2715 [hep-th]].
  %%CITATION = PHRVA,D81,065025;%%

%%  
%%  \bibitem{orco}
%%  E.~Witten,
%%  %``A Supersymmetric Form Of The Nonlinear Sigma Model In Two-Dimensions,''
%%  Phys.\ Rev.\  D {\bf 16}, 2991 (1977);
%%  %%CITATION = PHRVA,D16,2991;%%
%%  P.~Di Vecchia and S.~Ferrara,
%%  %``Classical Solutions In Two-Dimensional Supersymmetric Field Theories,''
%%  Nucl.\ Phys.\  B {\bf 130}, 93 (1977).
%%  %%CITATION = NUPHA,B130,93;%%
%%
%%  \bibitem{Bruno}
%%  B.~Zumino,
%%  %``Supersymmetry And Kahler Manifolds,''
%%  Phys.\ Lett.\  B {\bf 87}, 203 (1979).
%%  %%CITATION = PHLTA,B87,203;%%
%%
%%  \bibitem{rev1}
%%  V.~A.~Novikov, M.~A.~Shifman, A.~I.~Vainshtein and V.~I.~Zakharov,
%%  %``Two-Dimensional Sigma Models: Modeling Nonperturbative Effects Of Quantum
%%  %Chromodynamics,''
%%  Phys.\ Rept.\  {\bf 116}, 103 (1984).
%%  %%CITATION = PRPLC,116,103;%%
%%   
%%  \bibitem{rev2} 
%%  A.~M.~Perelomov,
%%  %``SUPERSYMMETRIC CHIRAL MODELS: GEOMETRICAL ASPECTS,''
%%  Phys.\ Rept.\  {\bf 174}, 229 (1989).
%%  %%CITATION = PRPLC,174,229;%%
%%  
%%  \bibitem{WI}
%%  E.~Witten,
%%  %``Constraints On Supersymmetry Breaking,''
%%  Nucl.\ Phys.\  B {\bf 202}, 253 (1982).
%%  %%CITATION = NUPHA,B202,253;%%
%%  
%%  \bibitem{twisted}
%%  L.~Alvarez-Gaum\'{e} and D.~Z.~Freedman,
%%  %``Potentials For The Supersymmetric Nonlinear Sigma Model,''
%%  Commun.\ Math.\ Phys.\  {\bf 91}, 87 (1983);
%%  %%CITATION = CMPHA,91,87;%%
%%  S.~J.~Gates,
%%  %``Superspace Formulation Of New Nonlinear Sigma Models,''
%%  Nucl.\ Phys.\ B {\bf 238}, 349 (1984);
%%  %%CITATION = NUPHA,B238,349;%%
%%  S.~J.~Gates, C.~M.~Hull and M.~Ro\v{c}ek,
%%  %``Twisted Multiplets And New Supersymmetric Nonlinear Sigma Models,''
%%  Nucl.\ Phys.\ B {\bf 248}, 157 (1984).
%%  %%CITATION = NUPHA,B248,157;%%
%%
%%  \bibitem{BelPo}
%%  A.~M.~Polyakov,
%%  %``Interaction Of Goldstone Particles In Two-Dimensions. Applications To
%%  %Ferromagnets And Massive Yang-Mills Fields,''
%%  Phys.\ Lett.\  B {\bf 59}, 79 (1975).
%%  %%CITATION = PHLTA,B59,79;%%
%%  
%%  \bibitem{adam}
%%  A.~Ritz, M.~Shifman and A.~Vainshtein,
%%  %``Counting domain walls in N = 1 super Yang-Mills,''
%%  Phys.\ Rev.\  D {\bf 66}, 065015 (2002)
%%  [arXiv:hep-th/0205083].
%%  %%CITATION = PHRVA,D66,065015;%%
%%  
%%  \bibitem{WessBagger}
%%  J. Wess and J. Bagger, {\em Supersymmetry and Supergravity}, Second Edition,
%%  Princeton University Press, 1992.
%%
%%  \bibitem{Helgason}
%%  S. Helgason, {\sl Differential geometry, Lie groups and symmetric spaces},
%%  Academic Press, New York, 1978.
%%  
%%  \bibitem{Dor}
%%  N.~Dorey,
%%  %``The BPS spectra of two-dimensional
%%  %supersymmetric gauge theories
%%  %with  twisted mass terms,''
%%  JHEP {\bf 9811}, 005 (1998) [hep-th/9806056].
%%  %%CITATION = HEP-TH 9806056;%%
%%
%%  \bibitem{Witten:2005px}
%%  E.~Witten,
%%  {\em Two-dimensional models with (0,2) supersymmetry: Perturbative aspects,}
%%  arXiv:hep-th/0504078.
%%  %%CITATION = HEP-TH/0504078;%%
%%  
%%  \bibitem{GSYphtr}
%%  A.~Gorsky, M.~Shifman and A.~Yung,
%%   %``Higgs and Coulomb/confining phases in "twisted-mass" deformed \cpn model,''
%%  Phys.\ Rev.\ D {\bf 73}, 065011 (2006)
%%  [hep-th/0512153].
%%
%%  \bibitem{Coleman}
%%  S.~R.~Coleman,
%%  %``More About The Massive Schwinger Model,''
%%  Annals Phys.  {\bf 101}, 239 (1976).
%%
%%  \bibitem{GSY05}
%%  A.~Gorsky, M.~Shifman and A.~Yung,
%%   %``Non-Abelian Meissner effect in Yang-Mills theories at weak
%%  %coupling,''
%%  Phys.\ Rev.\ D {\bf 71}, 045010 (2005)
%%  [hep-th/0412082].
%%  %%CITATION = HEP-TH 0412082;%%
%%
%%  \bibitem{Ferrari}
%%  F.~Ferrari,
%%  % ``LARGE N AND DOUBLE SCALING LIMITS IN TWO-DIMENSIONS.''
%%  JHEP {\bf 0205} 044 (2002)
%%  [hep-th/0202002].
%%
%%  \bibitem{Ferrari2}
%%  F.~Ferrari,
%%  %'' NONSUPERSYMMETRIC COUSINS OF SUPERSYMMETRIC GAUGE THEORIES:
%%  %QUANTUM SPACE OF PARAMETERS AND DOUBLE SCALING LIMITS.''
%%   Phys. Lett. {\bf B496} 212 (2000)
%%  [hep-th/0003142];
%%  %``A model for gauge theories with Higgs fields,''
%%  JHEP {\bf 0106}, 057 (2001)
%%  [hep-th/0102041].
%%  %%CITATION = HEP-TH 0102041;%%
%%
%%  \bibitem{AdDVecSal}
%%  A.~D'Adda, A.~C.~Davis, P.~DiVeccia and P.~Salamonson,
%%  %"An effective action for the supersymmetric CP$^{n-1}$ models,"
%%  Nucl.\ Phys.\ {\bf B222} 45 (1983).
%%
%%  \bibitem{ChVa}
%%  S.~Cecotti and C. Vafa,
%%  %"On classification of \ntwo supersymmetric theories,"
%%  Comm. \ Math. \ Phys. \ {\bf 158} 569 (1993)
%%  [hep-th/9211097].
%%
%%  \bibitem{HaHo}
%%  A.~Hanany and K.~Hori,
%%  %``Branes and N = 2 theories in two dimensions,''
%%  Nucl.\ Phys.\  B {\bf 513}, 119 (1998)
%%  [arXiv:hep-th/9707192].
%%  %%CITATION = NUPHA,B513,119;%%
%%
%%  \bibitem{AD}
%%  P. C.~Argyres and M. R.~Douglas,
%%  %``New Phenomena in SU(3) Supersymmetric Gauge Theory'' 
%%  Nucl. \ Phys. \ {\bf B448}, 93 (1995)   
%%  [arXiv:hep-th/9505062].
%%  %%CITATION = NUPHA,B448,93;%%
%%  
%%  \bibitem{APSW}
%%  P. C. Argyres, M. R. Plesser, N. Seiberg, and E. Witten,
%%  %``New N=2 Superconformal Field Theories in Four Dimensions''
%%  Nucl. \ Phys.  \ {\bf B461}, 71 (1996) 
%%  [arXiv:hep-th/9511154].
%%  %%CITATION = NUPHA,B461,71;%%
%%
%%  \bibitem{SYrev}
%%  M.~Shifman and A.~Yung,
%%  %{\sl Supersymmetric Solitons,}
%%  Rev.\ Mod.\ Phys. {\bf 79} 1139 (2007)
%%  [arXiv:hep-th/0703267].
%%  %%CITATION = HEP-TH/0703267;%%
%%
%%  \bibitem{Tonghetdyn}
%%  D.~Tong,
%%  %``The quantum dynamics of heterotic vortex strings,''
%%  JHEP {\bf 0709}, 022 (2007)
%%  [arXiv:hep-th/0703235].
%%  %%CITATION = JHEPA,0709,022;%%
%%  
%%  \bibitem{VYan}
%%  G.~Veneziano and S.~Yankielowicz,
%%  %``An Effective Lagrangian For The Pure N=1 Supersymmetric Yang-Mills
%%  %Theory,''
%%  Phys.\ Lett.\  B {\bf 113}, 231 (1982).
%%  %%CITATION = PHLTA,B113,231;%%
%%  
%%  \bibitem{ls}
%%  A.~Losev and M.~Shifman,
%%  %``N = 2 sigma model with twisted mass and superpotential: Central charges
%%  %and solitons,''
%%  Phys.\ Rev.\  D {\bf 68}, 045006 (2003)
%%  [arXiv:hep-th/0304003].
%%  %%CITATION = PHRVA,D68,045006;%%
%%  
%%  \bibitem{ls1}
%%  M.~Shifman, A.~Vainshtein and R.~Zwicky,
%%  %``Central charge anomalies in 2D sigma models with twisted mass,''
%%  J.\ Phys.\ A  {\bf 39}, 13005 (2006)
%%  [arXiv:hep-th/0602004].
%%  %%CITATION = JPAGB,A39,13005;%
%%  
%%  \bibitem{SYneww}
%%  M.~Shifman and A.~Yung,
%%  %``N=(0,2) Deformation of the N=(2,2) Wess-Zumino Model in Two Dimensions,''
%%  Phys.\ Rev.\  D {\bf 81}, 105022 (2010)
%%  [arXiv:0912.3836 [hep-th]].
%%  %%CITATION = PHRVA,D81,105022;%%
%%  
%%  \bibitem{D1}
%%  J.~Distler and S.~Kachru,
%%  %``(0,2) Landau-Ginzburg theory,''
%%  Nucl.\ Phys.\  B {\bf 413}, 213 (1994)
%%  [arXiv:hep-th/9309110].
%%  %%CITATION = NUPHA,B413,213;%%
%%  
%%  \bibitem{D2}
%%  T.~Kawai and K.~Mohri,
%%  %``Geometry Of (0,2) Landau-Ginzburg Orbifolds,''
%%  Nucl.\ Phys.\  B {\bf 425}, 191 (1994)
%%  [arXiv:hep-th/9402148].
%%  %%CITATION = NUPHA,B425,191;%%
%%  
%%  \bibitem{D3}
%%  I.~V.~Melnikov,
%%  %``(0,2) Landau-Ginzburg Models and Residues,''
%%  JHEP {\bf 0909}, 118 (2009)
%%  [arXiv:0902.3908 [hep-th]].
%%  %%CITATION = JHEPA,0909,118;%%
%%  
%%  \bibitem{Shifman:2009ay}
%%  M.~Shifman and A.~Yung,
%%  %``Crossover between Abelian and non-Abelian confinement in N=2 supersymmetric
%%  %QCD,''
%%  Phys.\ Rev.\  D {\bf 79}, 105006 (2009)
%%  [arXiv:0901.4144 [hep-th]].
%%  %%CITATION = PHRVA,D79,105006;%%
%%
%%  \bibitem{Tadpoint}
%%  D.~Tong,
%%  %``Superconformal  vortex strings,''
%%  JHEP {\bf 0612}, 051 (2006)
%%  [arXiv:hep-th/0610214].
%%
%%  \bibitem{SMMS}
%%  A.~Migdal and M.~Shifman,
%%  %``Dilaton Effective Lagrangian In Gluodynamics,''
%%  Phys.\ Lett.\  B {\bf 114}, 445 (1982).
%%  %%CITATION = PHLTA,B114,445;%%
%%  
%%  \bibitem{Kos}
%%  A.~Kovner and M.~A.~Shifman,
%%  %``Chirally symmetric phase of supersymmetric gluodynamics,''
%%  Phys.\ Rev.\  D {\bf 56}, 2396 (1997)
%%  [arXiv:hep-th/9702174].
%%  %%CITATION = PHRVA,D56,2396;%%
%%
\end{thebibliography}

\end{document}
