\documentclass[epsfig,12pt]{article}
\usepackage{epsfig}
\usepackage{graphicx}
\usepackage{rotating}
\usepackage{latexsym}
\usepackage{amsmath}
\usepackage{amssymb}
\usepackage{relsize}
\usepackage{geometry}
\geometry{letterpaper}
\usepackage{color}
\usepackage{bm}
\usepackage{slashed}
\usepackage{showlabels}




%%%%%%%%%%%%%%%%%%%%%%%%%%%%%%%%%%%%%%%%%%%%%%%%%%%%%%%%%%%%%%%%%%%%%%%%%%%%%%%%
%                                                                              %
%                                                                              %
%                     D O C U M E N T   S E T T I N G S                        %
%                                                                              %
%                                                                              %
%%%%%%%%%%%%%%%%%%%%%%%%%%%%%%%%%%%%%%%%%%%%%%%%%%%%%%%%%%%%%%%%%%%%%%%%%%%%%%%%
\def\baselinestretch{1.1}
\renewcommand{\theequation}{\thesection.\arabic{equation}}

\hyphenation{con-fi-ning}
\hyphenation{Cou-lomb}
\hyphenation{Yan-ki-e-lo-wicz}




%%%%%%%%%%%%%%%%%%%%%%%%%%%%%%%%%%%%%%%%%%%%%%%%%%%%%%%%%%%%%%%%%%%%%%%%%%%%%%%%
%                                                                              %
%                                                                              %
%                      C O M M O N   D E F I N I T I O N S                     %
%                                                                              %
%                                                                              %
%%%%%%%%%%%%%%%%%%%%%%%%%%%%%%%%%%%%%%%%%%%%%%%%%%%%%%%%%%%%%%%%%%%%%%%%%%%%%%%%
\def\beq{\begin{equation}}
\def\eeq{\end{equation}}
\def\beqn{\begin{eqnarray}}
\def\eeqn{\end{eqnarray}}
\def\beqn{\begin{eqnarray}}
\def\eeqn{\end{eqnarray}}
\def\nn{\nonumber}
\def\ba{\beq\new\begin{array}{c}}
\def\ea{\end{array}\eeq}
\def\be{\ba}
\def\ee{\ea}
\newcommand{\beas}{\begin{eqnarray*}}
\newcommand{\eeas}{\end{eqnarray*}}
\newcommand{\defi}{\stackrel{\rm def}{=}}
\newcommand{\non}{\nonumber}
\newcommand{\bquo}{\begin{quote}}
\newcommand{\enqu}{\end{quote}}
\newcommand{\m}{\tilde m}
\newcommand{\trho}{\tilde{\rho}}
\newcommand{\tn}{\tilde{n}}
\newcommand{\tN}{\tilde N}
%\newcommand{\p}{\partial}
\newcommand{\gsim}{\lower.7ex\hbox{$\;\stackrel{\textstyle>}{\sim}\;$}}
\newcommand{\lsim}{\lower.7ex\hbox{$\;\stackrel{\textstyle<}{\sim}\;$}}

%%%%%%%%%%%%%%%%%%%%%%%%%%%%%%%%%% definitions

\def\de{\partial}
\def\const{\hbox {\rm const.}}  
\def\o{\over}
\def\im{\hbox{\rm Im}}
\def\re{\hbox{\rm Re}}
\def\bra{\langle}\def\ket{\rangle}
\def\Arg{\hbox {\rm Arg}}
\def\Re{\hbox {\rm Re}}
\def\Im{\hbox {\rm Im}}
\def\diag{\hbox{\rm diag}}


%%%%%%%%%%%%%%%%%%%%%%%%%%%%%%%%%%%%%%%%%%%%%%%%%%%%%%%%%%%%%%%%%%%%

\def\QATOPD#1#2#3#4{{#3 \atopwithdelims#1#2 #4}}
\def\stackunder#1#2{\mathrel{\mathop{#2}\limits_{#1}}}
\def\stackreb#1#2{\mathrel{\mathop{#2}\limits_{#1}}}
\def\res{{\rm res}}
\def\Bf#1{\mbox{\boldmath $#1$}}
\def\balpha{{\Bf\alpha}}
\def\bbeta{{\Bf\beta}}
\def\bgamma{{\Bf\gamma}}
\def\bnu{{\Bf\nu}}
\def\bmu{{\Bf\mu}}
\def\bphi{{\Bf\phi}}
\def\bPhi{{\Bf\Phi}}
\def\bomega{{\Bf\omega}}
\def\blambda{{\Bf\lambda}}
\def\brho{{\Bf\rho}}
\def\bsigma{{\bfit\sigma}}
\def\bxi{{\Bf\xi}}
\def\bbeta{{\Bf\eta}}
\def\d{\partial}
\def\der#1#2{\frac{\d{#1}}{\d{#2}}}
\def\Im{{\rm Im}}
\def\Re{{\rm Re}}
\def\rank{{\rm rank}}
\def\diag{{\rm diag}}
\def\2{{1\over 2}}
\def\x{\stackrel{\otimes}{,}}

\def\ba{\beq\new\begin{array}{c}}
\def\ea{\end{array}\eeq}
\def\be{\ba}
\def\ee{\ea}
\def\stackreb#1#2{\mathrel{\mathop{#2}\limits_{#1}}}



\newcommand{\nfour}{${\cal N}=4\;$}
\newcommand{\none}{${\mathcal N}=1\,$}
\newcommand{\nonen}{${\mathcal N}=1$}
\newcommand{\ntwo}{${\mathcal N}=2$}
\newcommand{\ntt}{${\mathcal N}=(2,2)\,$}
\newcommand{\nzt}{${\mathcal N}=(0,2)\,$}
\newcommand{\ntwon}{${\mathcal N}=2$}
\newcommand{\ntwot}{${\mathcal N}= \left(2,2\right) $ }
\newcommand{\ntwoo}{${\mathcal N}= \left(0,2\right) $ }
\newcommand{\ntwoon}{${\mathcal N}= \left(0,2\right)$}


\newcommand{\ca}{{\mathcal A}}
\newcommand{\cell}{{\mathcal L}}
\newcommand{\cw}{{\mathcal W}}
\newcommand{\cs}{{\mathcal S}}
\newcommand{\vp}{\varphi}
\newcommand{\pt}{\partial}
\newcommand{\ve}{\varepsilon}
\newcommand{\gs}{g^{2}}
\newcommand{\zn}{$Z_N$}
\newcommand{\cd}{${\mathcal D}$}
\newcommand{\cde}{{\mathcal D}}
\newcommand{\cf}{${\mathcal F}$}
\newcommand{\cfe}{{\mathcal F}}
\newcommand{\ff}{\mc{F}}
\newcommand{\bff}{\ov{\mc{F}}}


\newcommand{\p}{\partial}
\newcommand{\wt}{\widetilde}
\newcommand{\ov}{\overline}
\newcommand{\mc}[1]{\mathcal{#1}}
\newcommand{\md}{\mathcal{D}}
\newcommand{\ml}{\mathcal{L}}
\newcommand{\mw}{\mathcal{W}}
\newcommand{\ma}{\mathcal{A}}


\newcommand{\GeV}{{\rm GeV}}
\newcommand{\eV}{{\rm eV}}
\newcommand{\Heff}{{\mathcal{H}_{\rm eff}}}
\newcommand{\Leff}{{\mathcal{L}_{\rm eff}}}
\newcommand{\el}{{\rm EM}}
\newcommand{\uflavor}{\mathbf{1}_{\rm flavor}}
\newcommand{\lgr}{\left\lgroup}
\newcommand{\rgr}{\right\rgroup}


\newcommand{\Mpl}{M_{\rm Pl}}
\newcommand{\suc}{{{\rm SU}_{\rm C}(3)}}
\newcommand{\sul}{{{\rm SU}_{\rm L}(2)}}
\newcommand{\sutw}{{\rm SU}(2)}
\newcommand{\suth}{{\rm SU}(3)}
\newcommand{\ue}{{\rm U}(1)}


\newcommand{\LN}{\Lambda_\text{SU($N$)}}
\newcommand{\sunu}{{\rm SU($N$) $\times$ U(1) }}
\newcommand{\sunun}{{\rm SU($N$) $\times$ U(1)}}
\def\cfl {$\text{SU($N$)}_{\rm C+F}$ }
\def\cfln {$\text{SU($N$)}_{\rm C+F}$}
\newcommand{\mUp}{m_{\rm U(1)}^{+}}
\newcommand{\mUm}{m_{\rm U(1)}^{-}}
\newcommand{\mNp}{m_\text{SU($N$)}^{+}}
\newcommand{\mNm}{m_\text{SU($N$)}^{-}}
\newcommand{\AU}{\mc{A}^{\rm U(1)}}
\newcommand{\AN}{\mc{A}^\text{SU($N$)}}
\newcommand{\aU}{a^{\rm U(1)}}
\newcommand{\aN}{a^\text{SU($N$)}}
\newcommand{\baU}{\ov{a}{}^{\rm U(1)}}
\newcommand{\baN}{\ov{a}{}^\text{SU($N$)}}
\newcommand{\lU}{\lambda^{\rm U(1)}}
\newcommand{\lN}{\lambda^\text{SU($N$)}}
\newcommand{\bxir}{\ov{\xi}{}_R}
\newcommand{\bxil}{\ov{\xi}{}_L}
\newcommand{\xir}{\xi_R}
\newcommand{\xil}{\xi_L}
\newcommand{\bzl}{\ov{\zeta}{}_L}
\newcommand{\bzr}{\ov{\zeta}{}_R}
\newcommand{\zr}{\zeta_R}
\newcommand{\zl}{\zeta_L}
\newcommand{\nbar}{\ov{n}}
\newcommand{\nnbar}{n\ov{n}}
\newcommand{\muU}{\mu_\text{U}}


\newcommand{\cpn}{CP$^{N-1}$\,}
\newcommand{\CPC}{CP($N-1$)$\times$C }
\newcommand{\CPCn}{CP($N-1$)$\times$C}


\newcommand{\lar}{\lambda_R}
\newcommand{\lal}{\lambda_L}
\newcommand{\larl}{\lambda_{R,L}}
\newcommand{\lalr}{\lambda_{L,R}}
\newcommand{\blar}{\ov{\lambda}{}_R}
\newcommand{\blal}{\ov{\lambda}{}_L}
\newcommand{\blarl}{\ov{\lambda}{}_{R,L}}
\newcommand{\blalr}{\ov{\lambda}{}_{L,R}}


\newcommand{\bpsi}{\ov{\psi}{}}


\newcommand{\hphi}{\hat\phi{}}
\newcommand{\hbphi}{\hat{\ov\phi}{}}
\newcommand{\hxi}{\hat\xi{}}
\newcommand{\hbxi}{\hat{\ov\xi}{}}
\newcommand{\hsigma}{\hat\sigma{}}
\newcommand{\hbsigma}{\hat{\ov\sigma}{}}
\newcommand{\hlambda}{\hat\lambda{}}
\newcommand{\hblambda}{\hat{\ov\lambda}{}}
\newcommand{\hz}{\hat z{}}
\newcommand{\hbz}{\hat{\ov z}{}}
\newcommand{\hzeta}{\hat\zeta{}}
\newcommand{\hbzeta}{\hat{\ov\zeta}{}}



\newcommand{\qt}{\wt{q}}
\newcommand{\bq}{\ov{q}}
\newcommand{\bqt}{\overline{\widetilde{q}}}


\newcommand{\eer}{\epsilon_R}
\newcommand{\eel}{\epsilon_L}
\newcommand{\eerl}{\epsilon_{R,L}}
\newcommand{\eelr}{\epsilon_{L,R}}
\newcommand{\beer}{\ov{\epsilon}{}_R}
\newcommand{\beel}{\ov{\epsilon}{}_L}
\newcommand{\beerl}{\ov{\epsilon}{}_{R,L}}
\newcommand{\beelr}{\ov{\epsilon}{}_{L,R}}


\newcommand{\bi}{{\bar \imath}}
\newcommand{\bj}{{\bar \jmath}}
\newcommand{\bk}{{\bar k}}
\newcommand{\bl}{{\bar l}}
\newcommand{\bmm}{{\bar m}}


\newcommand{\nz}{{n^{(0)}}}
\newcommand{\no}{{n^{(1)}}}
\newcommand{\bnz}{{\ov{n}{}^{(0)}}}
\newcommand{\bno}{{\ov{n}{}^{(1)}}}
\newcommand{\Dz}{{D^{(0)}}}
\newcommand{\Do}{{D^{(1)}}}
\newcommand{\bDz}{{\ov{D}{}^{(0)}}}
\newcommand{\bDo}{{\ov{D}{}^{(1)}}}
\newcommand{\sigz}{{\sigma^{(0)}}}
\newcommand{\sigo}{{\sigma^{(1)}}}
\newcommand{\bsigz}{{\ov{\sigma}{}^{(0)}}}
\newcommand{\bsigo}{{\ov{\sigma}{}^{(1)}}}


\newcommand{\rrenz}{{r_\text{ren}^{(0)}}}
\newcommand{\bren}{{\beta_\text{ren}}}


\newcommand{\Tr}{\text{Tr}}
\newcommand{\Ts}{\text{Ts}}
\newcommand{\dm}{\hat{{\scriptstyle \Delta} m}}
\newcommand{\dmdag}{\hat{{\scriptstyle \Delta} m}{}^\dag}
\newcommand{\mhat}{\widehat{m}}
\newcommand{\deltam}{{\scriptstyle \Delta} m}
\newcommand{\nvac}{\vec{n}{}_\text{vac}}


\newcommand{\ie}{{\it i.e.}~}
\newcommand{\eg}{{\it e.g.}~}
\newcommand{\ansatz}{{\it ansatz} }


\begin{document}

%%%%%%%%%%%%%%%%%%%%%%%%%%%%%%%%%%%%%%%%%%%%%%%%%%%%%%%%%%%%%%%%%%%%%%%%%%%%%%%%
%                                                                              %
%                                                                              %
%                            T I T L E   P A G E                               %
%                                                                              %
%                                                                              %
%%%%%%%%%%%%%%%%%%%%%%%%%%%%%%%%%%%%%%%%%%%%%%%%%%%%%%%%%%%%%%%%%%%%%%%%%%%%%%%%
\begin{titlepage}


\begin{flushright}
FTPI-MINN-XX/XX, UMN-TH-XXXX/XX\\
September 28/2014/DRAFT
\end{flushright}

\vspace{1.0cm}

\begin{center}
{  \Large \bf  Low-Energy Effective Action\\[1mm]
    of the Supersymmetric CP\boldmath{$(N-1)$} Model\\[3mm]
    in the Large-\boldmath{$N$} Limit}
\end{center}



\vspace{2mm}

\begin{center}

 {\large
 \bf   Pavel A.~Bolokhov$^{\,a}$,  Mikhail Shifman$^{\,b}$ and \bf Alexei Yung$^{\,\,b,c}$}
\end {center}

\begin{center}

$^a${\it Theoretical Physics Department, St.Petersburg State University, Ulyanovskaya~1, 
	 Peterhof, St.Petersburg, 198504, Russia}\\
$^b${\it  William I. Fine Theoretical Physics Institute,
University of Minnesota,
Minneapolis, MN 55455, USA}\\
$^{c}${\it Petersburg Nuclear Physics Institute, Gatchina, St. Petersburg
188300, Russia
}
\end{center}

\vspace{0.6cm}

\begin{center}
{\large\bf Abstract}
\end{center}

\hspace{0.3cm}

	Motivated by an old would-be paradox we found a solution of supersymmetric CP$(N-1)$ models in superfields.
	Our main target is the  K\"ahler potential, since the superpotential term was exactly known since early 1990s.
	To this end we used the large-$N$ expansion to 
	perform a supersymmetric calculation in the leading order in $1/N$. 
	The models considered are ${\mathcal N}= (2,2)$ basic CP$(N-1)$ model and  its nonminimal ${\mathcal N}= (0,2)$ 
	deformation various aspects of which are being actively studied since 2007. 
	We also extend the above models by adding twisted masses.
\vspace{2cm}


\end{titlepage}


%%%%%%%%%%%%%%%%%%%%%%%%%%%%%%%%%%%%%%%%%%%%%%%%%%%%%%%%%%%%%%%%%%%%%%%%%%%%%%%%
%                                                                              %
%                                                                              %
%                            I N T R O D U C T I O N                           %
%                                                                              %
%                                                                              %
%%%%%%%%%%%%%%%%%%%%%%%%%%%%%%%%%%%%%%%%%%%%%%%%%%%%%%%%%%%%%%%%%%%%%%%%%%%%%%%%
\section{Introduction}
\label{sec1}

	Supersymmetric and non-supersymmetric CP$(N-1)$ models in two dimensions are exactly solvable in the large-$N$ limit \cite{0,1p,SYhet}. 
	Moreover, the superpotential part can be found exactly (for any $N$) \cite{1,2,3} in terms of the so-called 
	twisted superfield $\Sigma$ (to be defined below) in the form
\beq
	{\mathcal L}_{\rm sp}    ~~=~~    \text{const} \cdot \!\lgr \int\, 
		d\theta_Rd\bar\theta_L \left( \sqrt{2}\Sigma\, \log \sqrt{2}\Sigma ~-~  
					      \sqrt{2}\Sigma \right) ~~+~~ \text{h.c.}\rgr\!.
\label{1}
\eeq
	For brevity below we will sometimes refer to above expression as the Witten superpotential.
	Equation (\ref{1})  encodes all information about the anomalies of the CP$(N-1)$ model. 
	In this sense it is akin to the Veneziano--Yankielowicz superpotential \cite{Veneziano} in ${\mathcal N}=1$ 
	super-Yang-Mills theory in four dimensions. 
	However, the Veneziano-Yankielowicz result admittedly presents an effective superpotential 
	suitable only for determination of the vacuum structure while (\ref{1}) presents the exact 
	superpotential part of the solution of the CP$(N-1)$ model.

	The scalar superpotential for the CP$(N-1)$ models in the large-$N$ limit was  found in \cite{SYhet,BSYhet} 
	(from a nonsupersymmetric calculation) to have the form
\beq
	V    ~~=~~    \frac{N}{4\pi}\,\bigg\{\Lambda^2 ~~+~~ 
			\big|\sqrt{2}\sigma\big|^2\Big(\log \frac{\big|\sqrt{2}\sigma\big|^2}{\Lambda^2}
					~-~ 1\Big)\bigg\}\,,
\label{2}
\eeq
	where $\sigma$ is the lowest component of $\Sigma$. In \cite{BSYhet} it was checked that the critical points of (\ref{1})
	coincide with the minima of the scalar potential (\ref{2}). Both expressions lead to zero-energy ground states, 
	{\it i.e.}  to supersymmetric vacua.

	Rather often an apparent paradox is pointed out in comparing (\ref{1}) and (\ref{2}). 
	Usually it is assumed that the K\"ahler term in the superfield formulation has the simplest possible form 
	invariant under scale transformations (and those related to the scale transformations by supersymmetry), namely,
\beq
	{\mathcal L}_\text{K}    ~~=~~    \text{const} \cdot  
				\int\, d^4 \theta\, \log \!\sqrt{2}\Sigma\,\,  \log\! \sqrt{2}\bar\Sigma\,.
\label{3}
\eeq
	Then, eliminating the $D$ term (the last component of $\Sigma$) by using equations of motion for $D$,
	we would obviously arrive at
\beq
	V    ~~=~~    \text{const} \cdot \left(\, \big|\sqrt{2}\sigma\big|^2\, \log \big|\sqrt{2}\sigma\big|^2 \,\right)^2\,.
\eeq
	It is clear that this expression, being proportional to the square of a logarithm, cannot coincide
	with (\ref{2}). 

	A way out from this paradox was pointed out in  \cite{BSYhet}. 
	There it was suggested that the minimal  form of the 
	K\"ahler term (\ref{3}) is not complete. 
	The K\"ahler term is in fact more complicated (although still compatible with scale invariance) ---  
	there is an extra contribution to ${\mathcal L}_{\rm K}$ having a special feature of vanishing at $D=0$, 
	{\it i.e.} in supersymmetric vacua. 
	However, the result for the full ${\mathcal L}_{\rm K}$ was not derived.
	Here we close this gap carrying out a supersymmetric calculation of both ${\mathcal L}_{\rm K}$ and 
	${\mathcal L}_{\rm sp}$ to the leading order in $1/N$. 
	While the latter coincides with (\ref{1}), as was expected,
	the expression for ${\mathcal L}_{\rm K}$ in the large-$N$ solution brings a surprise.\footnote{
	Note that unlike ${\mathcal L}_{\rm sp}$
	the K\"ahler potential cannot be exactly established on general grounds and requires an actual 
	and rather cumbersome calculation which, fortunately, can be performed to the  leading order in $1/N$.
	} The K\"ahler potential obtained depends not only on $\Sigma$ and $\overline\Sigma$, as is usually the case,  
	but also on superderivatives of these superfields.  
	This is the reason why scale invariance apparent in (\ref{3}) can be maintained in additional terms $\Delta {\mathcal L}_{\rm K}$.

	Combining our expression for the full  K\"ahler potential with (\ref{1}) we recover the scalar potential (\ref{2}). 

	The outline of the paper is as follows. 
	Section \ref{ssuper} is introductory.
	Here we introduce our basic notation, in particular, 
	twisted superfields\footnote{Further notations can be found in Appendix.}, formulate the problem in more detail, 
	and essentially ``guess'' the supersymmetric large-$N$ solution in terms of twisted superfields.
	In Section \ref{saction} we expand the superfield action in components in the two-derivative approximation.
	At the end of that section we derive the scalar potential (\ref{2}) from the superpotential (\ref{1}) 
	and the K\"ahler potential we found in our supersymmetric calculation based on the large-$N$ expansion.
	Section \ref{shet} treats the nonminimal heterotic deformation of CP$(N-1)$ models worked out in \cite{EdTo,SY1}, 
	see also \cite{SYhet,BSYhet,BSY1}, to the leading order in $1/N$. 
	In Sec. \ref{stwist} we repeat the procedure with non vanishing twisted masses introduced in a standard way. 
	Finally, Section \ref{sfinal} summarizes our results and conclusions.


%%%%%%%%%%%%%%%%%%%%%%%%%%%%%%%%%%%%%%%%%%%%%%%%%%%%%%%%%%%%%%%%%%%%%%%%%%%%%%%%
%                                                                              %
%                                                                              %
%                     S U P E R S Y M M E T R I Z A T I O N                    %
%                                                                              %
%                                                                              %
%%%%%%%%%%%%%%%%%%%%%%%%%%%%%%%%%%%%%%%%%%%%%%%%%%%%%%%%%%%%%%%%%%%%%%%%%%%%%%%%
\section{Supersymmetrization of the Effective Potential}
\label{ssuper}

	We start from the supersymmetric two-dimensional \cpn model,
\begin{align}
\label{sigma22}
% 
\notag
 	\mc{L}_\text{(2,2)} & ~~=~~
	\,\frac{1}{4e^2}\,F_{kl}^2  ~+~ \frac{1}{e^2} \left|\p_k \sigma\right|^2 
	~+~ \frac{1}{2e^2}\, D^2
	~+~ \frac{1}{e^2}\, \blar\, i\p_L\, \lar  ~+~  \frac{1}{e^2} \blal\, i\p_R\, \lal
	~+~
	\\[2mm]
%
	&
	~~+~~
	\left| \nabla n \right|^2  ~+~ | \sqrt{2}\sigma |^2 \left| n^l \right|^2
	~+~ iD \left( \left|n^l \right|^2 - 2\beta \right)
	~+~
	\\[2.8mm]
%
\notag	&
	~~+~~ \bxir\, i\nabla_L \xir  ~+~ \bxil\, i\nabla_R \xil ~+~
	i\sqrt{2}\sigma\, \ov{\xi}{}_{Rl} \xi_L^l
	~+~ i\sqrt{2}\ov{\sigma}\, \ov{\xi}{}_{Ll} \xi_R^l
	~+~
	\\[2.8mm]
%
\notag
	&
	~~+~~ i\sqrt{2}\, \ov{\xi_{[R}\, \lambda}{}_{L]}\, n
	~-~ i\sqrt{2}\, \nbar\,  \lambda_{[R}\, \xi_{L]}\,,
	\qquad
	l  ~=~  1,\,...\,N\,.
\end{align}
	The notations are explained in \cite{BSY3}, although they are of no great importance
	for our discussion. 
	In \cite{SYhet}, the large-$ N $ effective the theory was found
	by integrating over $ n^l $ fields and their superpartners $ \xi $:
\begin{align}
%
\notag
	\mc{L}_\text{eff}	& ~~=~~
                \frac{1}{4e_\gamma^2}\,F_{03}^2
		~~+~~ \frac{1}{e_{\sigma 1}^2}\,\big|\p_\mu\,\text{Re}\,\sigma\big|^2
		~~+~~ \frac{1}{e_{\sigma 2}^2}\,\big|\p_\mu\,\text{Im}\,\sigma\big|^2
		~~+~~
	\\[2mm]
%
\label{Leff}
	&
                ~~+~~ \frac{1}{e_\lambda^2}\,\ov{\lambda}{}_R\, i\p_L\, \lambda_R
                ~~+~~ \frac{1}{e_\lambda^2}\,\ov{\lambda}{}_L\, i\p_R\, \lambda_L
                ~~+~~ V_\text{eff}(D, \sigma)
                ~~+~~ ...
\end{align}
	where the effective potential is,
\beq
\label{Veff}
	V_\text{eff}    ~~=~~    -\, \frac{N}{4\pi} 
	\bigg\{
		\big( \big|\sqrt{2}\sigma\big|^2 \,+\, iD \big) 
		\log \big( \big|\sqrt{2}\sigma\big|^2 \,+\, iD \big)
		~-~
		iD
		~-~
		\big|\sqrt{2}\sigma\big|^2 \log \big|\sqrt{2}\sigma\big|^2
	\bigg\}\,.
\eeq
	We quickly note that expression \eqref{Veff} can be cast
	in integral forms which will appear useful further on in our paper,
\begin{align}
%
\notag
	\frac{4\pi}{N}\, V_\text{eff} &    ~~=~~    -\, \int_0^{iD} \!\!dt\, \ln \big(\, \big|\sqrt{2}\sigma\big|^2 \,+\, t \,\big)
	~~=
	\\
%
\label{Vint}
	&
	~~=~~    - \big|\sqrt{2}\sigma\big|^2 \lgr \ln \big|\sqrt{2}\sigma\big|^2 \cdot x ~+~
						\int_0^x \ln\, (1 + x)\, dx \rgr.
\end{align}
	Here $ x $ is a variable that is slightly more convenient to use than $ D $,
\beq
	x    ~~=~~    \frac{iD}{\big|\sqrt{2}\sigma\big|^2}\,.
\eeq
	The above expression is now easy to write also as a series in $ x $,
\beq
\label{Vser}
	 \frac{4\pi}{N}\, V_\text{eff}    ~~=~~
		x\, \big|\sqrt{2}\sigma\big|^2 \lgr -\, \ln\, \big|\sqrt{2}\sigma\big|^2 ~+~
		\sum_{k \geq 1}\, \frac{(-1)^k} 
                                      { k\,(k + 1) }\, x^k \rgr.
\eeq
	In this form, the potential precisely agrees with the effective action found in \cite{1p},
	in the limit where space-time derivatives of fields are discarded.

	The coupling constants in Eq.~\eqref{Leff} depend on $ D $ and $ \sigma $,
\begin{align}
%
\notag
	\frac{4\pi}{N}\, \frac{1}{e_\gamma^2}  & ~~=~~
                                        \frac{1}{3}\,\frac{1}{iD + \big|\sqrt{2}\sigma\big|^2}  ~+~
                                        \frac{2}{3}\,\frac{1}{\big|\sqrt{2}\sigma\big|^2}\,,
        \\[4mm]
%
\notag
	\frac{4\pi}{N}\, \frac{1}{e_\lambda^2}  & ~~=~~
                                        \frac{1}{\big|\sqrt{2}\sigma\big|^2}\, \frac{ x ~-~ \ln(1 + x) }{x^2}\,,
	\\[2mm]
%
\label{couplings}
	\frac{4\pi}{N}\, \frac{1}{e_{\sigma 1}^2}  & ~~=~~
		\frac{1}{\big|\sqrt{2}\sigma\big|^2}
			\lgr \frac 1 3 ~+~ \frac 2 3\, \frac{\big|\sqrt{2}\sigma|^4}
							{\big(\, \big|\sqrt{2}\sigma\big|^2 \,+\, iD \,\big)^2} \rgr,
	\\[3mm]
%
\notag
	\frac{4\pi}{N}\, \frac{1}{e_{\sigma 1}^2}  & ~~=~~
		\frac{1}{\big|\sqrt{2}\sigma\big|^2}\,.
\end{align}

	Our goal is to find a supersymmetric expression written explicitly in superfields,
	the {\it constant} bosonic part of which would coincide with \eqref{Vser}.

	Although both $ \sigma $ and $ D $ are part of the \ntwot supermultiplet $ V $,
	it is more convenient to work in terms of the {\it twisted} chiral superfield $ \Sigma $
	instead of $ V $,
\begin{align}
\label{defSigma}
%%
	\Sigma    & ~~=~~    \frac{i}{\sqrt 2}\, D_L\, \ov D{}_R\, V\,,
	&
	\ov \Sigma    & ~~=~~    \frac{i}{\sqrt 2}\, D_R\, \ov D{}_L\, V\,.
\end{align}
	A general definition of a twisted chiral superfield is that
\beq
	D_L\, \Sigma    ~~=~~    \ov D{}_R\, \Sigma    ~~=~~    0\,.
\eeq
	Field $ \sigma $ naturally comes as the lowest component of superfield $ \Sigma $
	defined as in \eqref{defSigma},
\beq
	\Sigma     ~~=~~    \sigma  ~~-~~  \sqrt{2}\, \theta_R \ov\lambda{}_L
				    ~~+~~  \sqrt{2}\, \ov\theta{}_L \lambda_R
				    ~~+~~  \sqrt{2}\, \theta_R \ov\theta{}_L \lgr D ~-~ i\, F_{03} \rgr.
\eeq
	So, essentially, all occurences of field $ \sigma $ in \eqref{Vser} can be replaced with $ \Sigma $.

	In order to ``supersymmetrize'' field $ D $ we need to get to higher components of $ \Sigma $.
	This is achieved by introducing a twisted chiral superfield $ S $,
\beq
	S    ~~=~~    \frac{i}{2}\,\ov D{}_R D_L \ln \sqrt{2}\ov\Sigma\,,
	\qquad\qquad
	\ov S    ~~=~~    \frac{i}{2}\, \ov D{}_L D_R \ln \sqrt{2}\Sigma\,.
\eeq
	Now it appears that the ratio $ x \,=\, iD/|\sigma|^2 $ is best supersymmetrized \cite{1p} by
	the ratio of $ S $ and $ \Sigma $,
\beq
	x    ~~=~~    \frac{iD}{\big|\sqrt{2}\sigma\big|^2}    ~~~~\longrightarrow~~~~
		\frac{S}{\sqrt{2}\Sigma}\,.
\eeq

	Using twisted superfields one can construct {\it twisted superpotentials}
\beq
	\int\, d^2\tilde\theta\, \wt{\cw}(\Sigma)    ~~=~~    \frac{1}{2}\,\ov D{}_L\, D_R\, \wt{\cw}(\Sigma)\Big|\,,
\eeq
	and regular $ d^4\theta $ integrals --- of course, provided there is something non-twisted chiral
	in the integrand.

	Armed with these definitions, we can now translate series \eqref{Vser} into
	the language of superfields.
	The first term in Eq.~\eqref{Vser} comes from the so-called Witten's superpotential,
\beq
	i \int\, d^2\tilde\theta 
	\lgr
		\sqrt{2}\Sigma\, \ln \sqrt{2}\Sigma  ~-~ \sqrt{2}\Sigma
	\rgr.
\eeq
	The series in \eqref{Vser} can be shown to reside in the highest component of
	the following expression,
\beq
	\frac{i}{2}\, 
	S\,
	\sum_{k \geq 1}\, \frac{    (-1)^k    }
                           {  k\,(k + 1)\,(k + 2)  } \lgr \frac{S}{\sqrt{2}\Sigma} \rgr^k.
\eeq
	This, actually, accounts for all terms in the series in Eq.~\eqref{Vser} but the first one.
	That one happens to be part of a separate, ``superkinetic'' term $ |\ln \Sigma\,|^2 $ ---
	the choice for such a name will be clear later.

	Altogether, the effective action in an explicitly supersymmetric form can be written as,
\begin{align}
%
\notag
	\frac{4\pi}{N}\,\cell &    ~~=~~
			-\,
			\int\, d^4\theta\, \frac{1}{2}\, \Big| \ln \sqrt{2}\Sigma \, \Big|^2
			~~-~~
			i \int\, d^2\tilde\theta 
			\lgr
			\sqrt{2}\Sigma\, \ln \sqrt{2}\Sigma  ~-~ \sqrt{2}\Sigma
			\rgr
			~~+
	\\
%
\label{sseries}
	&
			\phantom{~~=~~}
			+~~
			\frac{i}{2}\, 
			\int\, d^2\tilde\theta\,
			S\,
			\sum_{k \geq 1}\, \frac{    (-1)^k    }
                                           {  k\,(k + 1)\,(k + 2)  } \lgr \frac{S}{\sqrt{2}\Sigma} \rgr^k
			~~+~~ \text{h.c.}.
\end{align}

	The series in the second line can be written as a $ D $-term.
	We get
\begin{align}
%
\notag
	\frac{4\pi}{N}\, \cell &    ~~=~~     
			-\,
			\int\, d^4\theta\, \frac{1}{2}\, \Big| \ln \sqrt{2}\Sigma \, \Big|^2
			~~-~~
			i \int\, d^2\tilde\theta 
			\lgr
			\sqrt{2}\Sigma\, \ln \sqrt{2}\Sigma  ~-~ \sqrt{2}\Sigma
			\rgr
			~~+
	\\
%
	&
			\phantom{~~=~~}
			+~~ 
			\frac{1}{2} \int\, d^4\theta\,
			\ln \sqrt{2}\ov\Sigma\,
			\sum_{k \geq 1}\, \frac{    (-1)^k    }
                                           {  k\,(k + 1)\,(k + 2)  }\, \lgr \frac{S}{\sqrt{2}\Sigma} \rgr^k
	~~+~~ \text{h.c.}
\end{align}
	Here we can see explicitly that the only $ F $-term is the Witten's superpotential,
	while all other terms are $ D $-terms.
	Finally, the above series obviously can be re-summed, 
	and not quite surprisingly will form a logarithm,
	as we started from a logarithm in Eq.~\eqref{Veff},
\begin{align}
%
\notag
	&
	\frac{4\pi}{N}\, \cell    ~~=~~     
			-\,
			\int\, d^4\theta\, \frac{1}{2}\, \Big| \ln \sqrt{2}\Sigma \, \Big|^2
			~~-~~
			i \int\, d^2\tilde\theta 
			\lgr
			\sqrt{2}\Sigma\, \ln \sqrt{2}\Sigma  ~-~ \sqrt{2}\Sigma
			\rgr
			~~-
	\\
%
\label{Lsuper}
			&
			-~~ 
			\frac{1}{4} \int\, d^4\theta\,
			\ln\, \sqrt{2}\ov\Sigma
			\lgr \Big( 1 \,+\, \frac{\sqrt{2}\Sigma}{S} \Big)^2\,
				\ln \Big(\, 1 \,+\, \frac{S}{\sqrt{2}\Sigma} \,\Big) ~-~
				\frac{\sqrt{2}\Sigma}{S} \rgr\!\!
			~~+~~ \text{h.c.}
\end{align}

	The above expression consists of three parts --- the ``superkinetic'' term, Witten's superpotential
	and an additional logarithmic $ D $-term.
	Both the superkinetic and the logarithmic $ D $-terms form the K\"{a}hler potential of the target space.
	For a minute, let us pretend that the complicated logarithmic $ D $-term did not exist, 
	and let us find the component expansion of the superkinetic term and Witten's potential,
\begin{align}
%
\notag
	\frac{4\pi}{N}\, \cell_{(0)} &    ~~=~~
	-\, \int\, d^4\theta\, \bigg\lgroup \frac{1}{2}\, \Big| \ln \sqrt{2}\Sigma \, \Big|^2
			~+~ 
	\\[2mm]
%
\notag
	&
			~+~
			i\, d^2\tilde\theta 
			\lgr \sqrt{2}\Sigma\, \ln \sqrt{2}\Sigma  ~-~ \sqrt{2}\Sigma \rgr
			~+~
			i\, d^2\ov{\tilde\theta}
			\lgr \sqrt{2}\ov\Sigma\, \ln \sqrt{2}\ov\Sigma  ~-~ \sqrt{2}\ov\Sigma \rgr
			\!\!\bigg\rgroup
			=~ 
	\\[2mm]
%
\notag
	&     ~~=~~
	-\,iD\, \log |\sqrt{2}\sigma|^2  ~~-~~  \frac{1}{2}\, \frac{ \big(i D\big)^2 } { \big|\sqrt{2}\sigma\big|^2 }
	~~-~~ F_{03}\, \log\, \frac{\sqrt{2}\sigma}{\sqrt{2}\ov\sigma}
	~~+~~
	\\[2mm]
%
\label{L0}
	&
	~~+~~ \frac{
		\big|\p_\mu \sigma\big|^2  ~+~  \frac{\displaystyle 1}{\displaystyle 2}\,F_{03}^2  ~+~
		\frac{\displaystyle 1}{\displaystyle 2}\!
			    \lgr\, \ov\lambda{}_R\, i\overleftrightarrow{\md_L} \lambda_R  ~+~ 
				   \ov\lambda{}_L\, i\overleftrightarrow{\md_R} \lambda_L \rgr
		} { \big|\sqrt{2}\sigma\big|^2 }
	~~-~~
	\\[2mm]
%
\notag
	&
	~~-~~ 2\,\frac{
			i\sqrt{2}\sigma\ov\lambda{}_R\lambda_L  ~+~  
			i\sqrt{2}\,\ov{\sigma\lambda}{}_L\lambda_R
		} { \big|\sqrt{2}\sigma\big|^2 }
	~~+~~ 2\, \frac{
			\ov\lambda{}_R \lambda_L \ov\lambda{}_L \lambda_R
		} { \big|\sqrt{2}\sigma\big|^4 }
	~~+~~
	\\[2mm]
%
\notag
	&
	~~+~~
	\frac{	(iD + F_{03})\, i\sqrt{2}\sigma \ov\lambda{}_R\lambda_L ~+~
		(iD - F_{03})\, i\sqrt{2}\,\ov{\sigma \lambda}{}_L\lambda_R  }
		{ \big|\sqrt{2}\sigma\big|^4 }\,.
\end{align}
	The long derivatives here are
\begin{align}
%
\notag
	\md_L &    ~~=~~    \p_L  ~~-~~  \p_L \ln \sqrt{2}\sigma\,,
	\\[1mm]
%
	\md_R &    ~~=~~    \p_R  ~~-~~  \p_R \ln \sqrt{2}\ov\sigma\,,
\end{align}
	with conjugate expressions for the left-ward derivatives 
	$ \overleftarrow{\md}{}_L $ and $ \overleftarrow{\md}{}_R $.
	These derivatives include the Christoffel symbol $ 1 / \sigma $ which arises from
	the fact that we have a  K\"ahler potential $ \big| \log \Sigma \big|^2 $.
	This potential is singular at zero, as the $ n^l $ fields of the 
	gauge formulation of the sigma model 
	(that were integrated out in \cite{SYhet} to find \eqref{Veff}) 
	would become massless at $ \sigma \,=\, 0 $.
	In coordinates $ \Sigma $, $ \ov\Sigma $ the K\"ahler potential generates a metric $ 1 / |\sigma|^2 $
	which is seen in the denominators in the expressions above.
	However, the Riemann tensor is zero for this metric, reflecting the
	observation that the target space is in fact flat ---
	if one adopts $ \log \Sigma $ and $ \log \ov\Sigma $ as the coordinates, the metric
	disappears completely.
	This also explains why the terms in Eq.~\eqref{L0} that come from the
	superkinetic term are scale invariant ---
	the logarithmic field $ \log \Sigma $ depends on scale-invariant ratios
	$ \ov\lambda{}_L/\sigma $, $ \lambda_R/\sigma $ {\it etc}.
	Still, we prefer to work with the coordinates $ \Sigma $ and $ \ov\Sigma $.

	The first two terms in the component expansion in \eqref{L0} form the 
	leading part of the effective potential \eqref{Vser}.
	The third term is the anomaly{\it, may need more discussion}.
	The terms on the fourth line in \eqref{L0} are the kinetic terms, 
	while the terms on the rest of the lines are Yukawa-like and quartic couplings.
	Although we have only written the ``trivial'' part of the action,
	it already contains quite a bit of low-energy information about the theory.
	We will now see that the complicated logarithmic $ D $-term gives a
	correction to Eq.~\eqref{L0}.


%%%%%%%%%%%%%%%%%%%%%%%%%%%%%%%%%%%%%%%%%%%%%%%%%%%%%%%%%%%%%%%%%%%%%%%%%%%%%%%%
%                                                                              %
%                                                                              %
%     E F F E C T I V E   A C T I O N   T O   T W O   D E R I V A T I V E S    %
%                                                                              %
%                                                                              %
%%%%%%%%%%%%%%%%%%%%%%%%%%%%%%%%%%%%%%%%%%%%%%%%%%%%%%%%%%%%%%%%%%%%%%%%%%%%%%%%
\section{Effective Action up to Two Derivatives}
\label{saction}
	Full component expansion of the effective action \eqref{Lsuper} is given
	in Appendix~\ref{app:expansion}.
	However, since at low energies we are only interested in the lowest level
	of momenta, we can limit ourselves to only the lowest powers of 
	space-time derivatives --- second for bosons and first for fermions.

	To obtain the low energy action in the two-derivative approximation,
	it is more convenient to work with the series representation \eqref{sseries}.
	We perform the superspace integration on each of the terms in the series,
	drop everything beyond one or two derivatives as needed,
	and then assemble the resulting terms back into logarithms.
	We find,
\begin{align}
%
\notag
	\frac{4\pi}{N}\, \cell_\text{two deriv} &    ~~=~~  
	\frac{ \big|\p_\mu \sigma\big|^2 }
	{ \big|\sqrt{2}\sigma\big|^2 }
	~~-~~
	F_{03}\, \log\, \frac{\sqrt{2}\sigma}{\sqrt{2}\ov\sigma}
	~~+~~
	\frac{4\pi}{N}\, V_\text{eff}
	~~-
	\\[2mm]
%
\notag
	&
	~~-~~
	\frac{F_{03}^2}{\big|\sqrt{2}\sigma\big|^2}
	\lgr 2\, \frac{ \ln (1 + x) \,-\, x } { x^2 }  ~+~
		\frac{1}{2}\, \frac{1}{1 \,+\, x} \rgr
	~~-
	\\[2mm]
%
\notag
	&
	~~-~~
	\frac{
		\ov\lambda{}_R\, i\overleftrightarrow{\md_L} \lambda_R  ~+~ 
		\ov\lambda{}_L\, i\overleftrightarrow{\md_R} \lambda_L
	} { \big|\sqrt{2}\sigma\big|^2 }
	\,
	\frac{ \ln (1 + x) \,-\, x } { x^2 }
	~~-
	\\[2mm]
%
\label{L2d}
	&
	~~-~~ 
	2\,\frac{
		i\sqrt{2}\sigma\ov\lambda{}_R\lambda_L  ~+~  
		i\sqrt{2}\,\ov{\sigma\lambda}{}_L\lambda_R
	} { \big|\sqrt{2}\sigma\big|^2 }\,
	\frac{ \ln (1 + x) } { x }
	~~+
	\\[2mm]
%
\notag
	&
	~~+~~
	4\,\frac{
		\ov\lambda{}_R \lambda_L \ov\lambda{}_L \lambda_R
	} { \big|\sqrt{2}\sigma\big|^4 }\,
	\lgr \frac{ \ln (1 + x) \,-\, x } { x^2 }  ~+~
		\frac{1}{1 \,+\, x} \rgr
	~~+
	\\[2mm]
%
\notag
	&
	~~+~~
	\frac{1}{4}\,\Box\,\log |\sqrt{2}\sigma|^2 \cdot
	\frac{ (1 - x^2) \ln (1 + x) \,-\, x } { x^2 }
	~~-
	\\[3mm]
%
\notag
	&
	~~-~~
	2\,\frac{ F_{03} \big( i\sqrt{2}\sigma\ov\lambda{}_R\lambda_L ~-~
			       i\sqrt{2}\,\ov{\sigma\lambda}{}_L\lambda_R \big) }
		{ \big|\sqrt{2}\sigma\big|^4 }\,
	\frac{ \ln (1 + x) \,-\, x } { x^2 }\,.
\end{align}
	Comparing Eqs \eqref{L2d} and \eqref{L0} we notice that the effect of adding 
	the supersymmetric series to Eq.~\eqref{L0} is the multiplication of all terms 
	by certain functions of $ x $.
	In fact, the whole expression \eqref{L0}
	is the leading order approximation (or subleading for some of the terms) in $ x $ of 
	the two-derivative Langrangian \eqref{L2d}.

	Now we exclude the auxiliary field $ D $ to obtain the effective action 
	as a function of $ \sigma $ only.
	Due to the very complicated dependence of the action \eqref{L2d} on $ D $, we can only do so 
	with the accuracy of keeping up to two space-time derivatives.
	In order to make the procedure of resolution of $ D $ easier, we switch to using the variable $ x $.
	The Lagrangian \eqref{L2d} can be split into three pieces:
\beq
\label{Lpert}
	\cell    ~~=~~    \cell_{(0)}(x)    ~~+~~    \cell_{(1)}(x)    ~~+~~    \cell_{(2)}(x)\,,
\eeq
	each explicitly containing, correspondingly, none, one and two space-time derivatives.
	In particular, 
\beq
	\frac{4\pi}{N}\, \cell_{(0)}(x)    ~~=~~
	\frac{ \big|\p_\mu \sigma\big|^2 }
	{ \big|\sqrt{2}\sigma\big|^2 }
	~~-~~
	F_{03}\, \log\, \frac{\sqrt{2}\sigma}{\sqrt{2}\ov\sigma}
	~~+~~
	\frac{4\pi}{N}\, V_\text{eff}(x)\,.
\eeq
	Although this expression actually does contain space-time derivatives in the kinetic term for $ \sigma $
	and in the anomaly term, the latter terms are independent of $ x $ and are unaltered during the
	course of resolution of $ x $.
	The first-order part $ \cell_{(1)}(x) $ only includes the term 
	on the fourth line in \eqref{L2d}, {\it i.e.} the Yukawa-like coupling.
	Each part in the expansion \eqref{Lpert}, besides the explicit derivatives, will also contain derivatives
	implictly, via $ x $.
	More specifically, we split the solution (to be found) of the equations of motion as,
\beq
	x    ~~=~~    x_0    ~~+~~    x_1    ~~+~~    ...\,.
\eeq
	Expanding the Lagrangian perturbatively in the number of space-time derivatives, and using the equation of motion for $ x $,
	one can show that to the second order in space-time derivatives the Lagrangian is
\beq
\label{Lx}
	\cell_\text{two deriv}(\sigma)    ~~=~~    
		\cell_{(0)}(x_0)    ~~+~~    \cell_{(1)}(x_0)    ~~+~~    \cell_{(2)}(x_0)    
	~~-~~	\frac{1}{2}\, \frac{\p^2 \cell_{(0)}}{\p x^2}\bigg|_{x_0} \!\!\cdot x_1^2\,.
\eeq
	Here $ x_0 $ is the minimum of the potential $ V_\text{eff}(x) $, 
\beq
	x_0    ~~=~~    \frac{ 1 \,-\, \big|\sqrt{2}\sigma\big|^2 }{ \big|\sqrt{2}\sigma\big|^2 }\,,
\eeq
	while $ x_1 $ is
\beq
	x_1    ~~=~~    -\, \frac{\p \cell_{(1)}}{\p x}(x_0) \lgr \frac{\p^2 \cell_{(0)}}{\p x^2}(x_0)\rgr^{-1}.
\eeq
	Higher terms in $ x $ are not needed for our purposes.
	It is only due to the presence of the Yukawa-like coupling in Eq.~\eqref{L2d} --- which effectively is 
	a one-derivative term --- that the expression \eqref{Lx} includes the last term depending on $ x_1^2 $.
	That term, however, is important as it modifies the coefficient of the quartic fermionic coupling in Eq.~\eqref{L2d}.

	It is now straightforward to calculate expression \eqref{Lx}. 
	For the sake of presenting the results, it is helpful to introduce a variable $ r $,
\beq
	r    ~~=~~    \big|\sqrt{2}\sigma\big|^2
\eeq
	and a function $ v_\text{eff}(r) $,
\beq
	v_\text{eff}(r)    ~~=~~    \frac{4\pi}{N}\, V_\text{eff}(\sigma)    ~~=~~    r\, \ln r  ~~+~~  1  ~~-~~  r\,.
\eeq
	The result for Eq.~\eqref{Lx} can now be given as,
\begin{align}
%
\notag
	& \frac{4\pi}{N}\, \cell_\text{two deriv}(\sigma, A_\mu, \lambda)    ~~=~~  
	\\[2mm]
%
\notag
	& ~~=~~
	\frac{ \big|\p_\mu \sigma\big|^2 } r
	~~-~~
	F_{03}\, \log\, \frac{\sqrt{2}\sigma}{\sqrt{2}\ov\sigma}
	~~+~~
	v_\text{eff}(r)
	~~-
	\\[2mm]
%
\notag
	&
	~~+~~
	2\, F_{03}^2
	\lgr \frac{v_\text{eff}(r)}{(1 - r)^2}  ~-~  \frac{r}{4} \rgr
	~~+~~
	\lgr 
		\ov\lambda{}_R\, i\overleftrightarrow{\md_L} \lambda_R  ~+~ 
		\ov\lambda{}_L\, i\overleftrightarrow{\md_R} \lambda_L
	\rgr
	\frac{v_\text{eff}(r)}{(1 - r)^2}
	~~+
	\\[2mm]
%
\label{Lsigma}
	&
	~~+~~ 
	2 \lgr
		i\sqrt{2}\sigma\ov\lambda{}_R\lambda_L  ~+~  
		i\sqrt{2}\,\ov{\sigma\lambda}{}_L\lambda_R
	\rgr
	\frac{1 - r}{r}\, \ln r
	~~-
	\\[2mm]
%
\notag
	&
	~~-~~
	4\, \ov\lambda{}_R \lambda_L \ov\lambda{}_L \lambda_R
	\lgr
		\frac{v_\text{eff}(r)}{r\, (1 - r)^2}
		~-~  \frac{1}{r}
		~+~  \frac{r\, \big( 1 \,-\, r \,-\, \ln r \big)^2}{ (1 - r)^4 }
	\rgr
	~-
	\\[2mm]
%
\notag
	&
	~~-~~
	\frac{1}{4}\,\Box\,\ln r
	\lgr
		\frac{r\, v_\text{eff}(r)}{(1 - r)^2}  ~-~  \ln r
	\rgr
	~~+
	\\[2mm]
%
\notag
	&
	~~+~~
	2\, F_{03} \lgr 
			i\sqrt{2}\sigma\ov\lambda{}_R\lambda_L ~-~
			i\sqrt{2}\,\ov{\sigma\lambda}{}_L\lambda_R
		   \rgr
	\frac{v_\text{eff}(r)}{r\, (1 - r)^2}\,.
\end{align}
	Interestingly, most of the ``coefficients'' of the terms in the Lagrangian \eqref{Lsigma} contain 
	the potential $ v_\text{eff}(r) $ in one or another form.


%%%%%%%%%%%%%%%%%%%%%%%%%%%%%%%%%%%%%%%%%%%%%%%%%%%%%%%%%%%%%%%%%%%%%%%%%%%%%%%%
%                                                                              %
%                                                                              %
%                   H E T E R O T I C   D E F O R M A T I O N                  %
%                                                                              %
%                                                                              %
%%%%%%%%%%%%%%%%%%%%%%%%%%%%%%%%%%%%%%%%%%%%%%%%%%%%%%%%%%%%%%%%%%%%%%%%%%%%%%%%
\section{Heterotic Deformation}
\label{shet}

	The heterotic deformation is introduced using \ntwoo superfields and
	essentially follows the guidelines of \cite{EdTo}.
	Before describing this deformation in detail we need to introduce the necessary
	superfield machinery consistent with our notations.


%%%%%%%%%%%%%%%%%%%%%%%%%%%%%%%%%%%%%%%%%%%%%%%%%%%%%%%%%%%%%%%%%%%%%%%%%%%%%%%%
%                                                                              %
%                     N = ( 0 , 2 )   S U P E R F I E L D S                    %
%                                                                              %
%%%%%%%%%%%%%%%%%%%%%%%%%%%%%%%%%%%%%%%%%%%%%%%%%%%%%%%%%%%%%%%%%%%%%%%%%%%%%%%%
\subsection{\boldmath{\ntwoo} superfields}

	We define \ntwoo superspace via reduction of \ntwot superspace by putting
\beq
	\theta_L    ~~=~~    \ov\theta{}_L    ~~=~~    0\,.
\eeq
	Each chiral and twisted-chiral \ntwot superfield this way splits into two \ntwoo superfields.
	While chiral superfields are usually described as being dependent on
	a ``holomorphic'' variable $ y^\mu $,
\beq
	y^\mu    ~~=~~    x^\mu    ~~+~~    i\,\ov{\theta\sigma}{}_\mu\theta\,,
\eeq
	and twisted-chiral as dependent on a twisted variable $ \wt y{}^\mu $,
\beq
	\wt y{}^\mu    ~~=~~    x^\mu    ~~+~~    i\,\ov{\theta\wt\sigma}{}_\mu\theta\,,
\eeq
	(see Appendix~\ref{app:notations} for details and notations), 
	the distinction between the two variables vanishes upon reduction to \ntwoo superspace.
	As a result, \ntwoo superspace defines only one kind of holomorphic superfields --- 
	chiral \ntwoo superfields, which depend upon the reduced variables $ \upsilon^\mu $
\begin{align}
%
\notag
	\upsilon^0 &    ~~=~~    x^0  ~~+~~  i\,\ov\theta{}_R\theta_R
	\\[2mm]
%
	\upsilon^3 &    ~~=~~    x^3  ~~+~~  i\,\ov\theta{}_R\theta_R\,.
\end{align}

	To make a distinction with \ntwot superfields, we denote the \ntwoo superfields
	by symbols with a hat on top --- $ \hsigma $, $ \hxi $, {\it etc}.
	Of immediate interest to us is the {\it positional} superfield $ Z(y) $ which splits
	into two \ntwoo superfields --- $ \hz(\upsilon) $ and $ \hzeta(\upsilon) $,
\begin{align}
%
\notag
	\hz(\upsilon) &    ~~=~~    z  ~~-~~  \sqrt{2}\,\theta_R\,\zeta_L\,,
	&
	\hzeta(\upsilon) &    ~~=~~    \zeta_R  ~~+~~  \sqrt{2}\,\theta_R\,\cfe\,,
	\\[2mm]
%
	\hbz(\ov\upsilon) &    ~~=~~    \ov z  ~~+~~  \sqrt{2}\,\ov{\theta_L \zeta}{}_L\,,
	&
	\hbzeta(\ov\upsilon) &    ~~=~~    \ov\psi{}_R  ~~+~~  \sqrt{2}\,\ov{\theta_R \cfe}\,,
\end{align}
	and the twisted chiral superfield $ \Sigma(\wt y) $, which splits into 
	$ \hsigma(\upsilon) $ and $ \hlambda(\upsilon) $,
\begin{align}
%
\notag
	\hsigma(\upsilon) &    ~~=~~    \sigma  ~~-~~  \sqrt{2}\,\theta_R\,\ov\lambda{}_L
	&
	\hlambda(\upsilon) &    ~~=~~    \lambda_R  ~~+~~  i\,\theta_R\,(iD \,+\, F_{03})
	\\[2mm]
%
	\hbsigma(\ov\upsilon) &    ~~=~~    \ov\sigma  ~~+~~  \sqrt{2}\,\ov\theta{}_R\,\lambda_L
	&
	\hblambda(\ov\upsilon) &    ~~=~~    \ov\lambda{}_R  ~~+~~  i\,\ov\theta{}_R\,(iD \,-\, F_{03})\,.
\end{align}


%%%%%%%%%%%%%%%%%%%%%%%%%%%%%%%%%%%%%%%%%%%%%%%%%%%%%%%%%%%%%%%%%%%%%%%%%%%%%%%%
%                                                                              %
%         C O N S T R U C T I O N   O F   T H E   D E F O R M A T I O N        %
%                                                                              %
%%%%%%%%%%%%%%%%%%%%%%%%%%%%%%%%%%%%%%%%%%%%%%%%%%%%%%%%%%%%%%%%%%%%%%%%%%%%%%%%
\subsection{Construction of the deformation}

	The heterotic deformation is introduced using the fermionic translational degree
	of freedom $ \zeta_R $, which, as we just described, sits in the 
	supermultiplet $ \hzeta $.
	The first thing this new degree of freedom needs is the kinetic term
\beq
	\int\, d^2\theta_R\,\, \hzeta_R\, \hbzeta_R    ~~=~~    \ov\zeta{}_R\,i\p_L \zeta_R  ~~+~~  \ov\cfe\,\cfe\,.
\eeq

	In the \ntwoo space, a superpotential, as a holomorphic function $ J(\hsigma) $ is constructed 
	using an arbitrary fermionic multiplet. It is obvious that for our purposes $ \hzeta $ is the
	right fermionic superfield, giving rise to terms like
\beq
	\int d\theta_R\,\, \hzeta\, J(\hsigma)\,.
\eeq
	As for the superpotential function $ J(\hsigma) $ itself, it comes from the four-dimensional
	deformation superpotential --- if applicable --- taken as a function of $ \hsigma $,
\beq
	J(\hsigma)    ~~=~~    \frac{\p\, \cw_\text{4-d}(\hsigma)}{\p \hsigma}\,.
\eeq
	Remind, that originally $ \cw_\text{4-d} $ is a function of $ \ca $. 
	We stick to the quadratic deformation, generically without making references to the four-dimensional
	bulk theory, in which case
\beq
	J(\sqrt{2}\hsigma)    ~~=~~    \delta\cdot \sqrt{2}\hsigma\,,
\eeq
	where $ \delta $ is the parameter of deformation.

	Altogether, the part of the Lagrangian involving the supertranslational sector is
\beq
	\cell_\text{het}    ~~=~~
	\int\, d^2\theta_R\,\, \hzeta_R\, \hbzeta_R
	~~-~~
	i\, \int d\theta_R\,\, \hzeta\cdot J(\sqrt{2}\hsigma)
	~~-~~
	i\, \int d\ov\theta{}_R\,\, \hbzeta\cdot \ov J(\sqrt{2}\hbsigma)\,.
\eeq
	In components, this is
\begin{align}
%
\notag
	\cell_\text{het} &   ~~=~~
		\ov\zeta{}_R\, i\p_L \zeta_R  ~~+~~  \ov\cfe\,\cfe  
		~~-~~  i\,\sqrt{2}\delta\, \cfe \cdot \sqrt{2}\sigma
		~~-~~  i\,\sqrt{2}\ov\delta\, \ov\cfe \cdot \sqrt{2}\ov\sigma
		~~+~~
	\\[2mm]
%
		&
		~~+~~  i\,\sqrt{2}\delta \cdot \sqrt{2}\ov\lambda{}_L \zeta_R
		~~+~~  i\,\sqrt{2}\ov\delta \cdot \sqrt{2}\ov\zeta{}_R \lambda_L\,.
\end{align}
	With the auxiliary field $ \cfe $ excluded, this produces a quadratic potential
	for $ \sigma $,
\beq
	\cell_\text{het}   ~~=~~
		\ov\zeta{}_R\, i\p_L \zeta_R
		~~+~~  |\sqrt{2}\delta|^2\, \big|\sqrt{2}\sigma\big|^2
		~~+~~  i\,\sqrt{2}\delta \cdot \sqrt{2}\ov\lambda{}_L \zeta_R
		~~+~~  i\,\sqrt{2}\ov\delta \cdot \sqrt{2}\ov\zeta{}_R \lambda_L\,.
\eeq
	Since the introduction of this deformation did not involve
	the variables $ n^l $ or $ \xi^l $ of the original \cpn theory,
	the above terms are just added on top of \eqref{Lsigma}
	(paying due respect to the factor $ N/4\pi $ in the latter equation) to produce
	the solution of the deformed theory.


%%%%%%%%%%%%%%%%%%%%%%%%%%%%%%%%%%%%%%%%%%%%%%%%%%%%%%%%%%%%%%%%%%%%%%%%%%%%%%%%
%                                                                              %
%                                                                              %
%                          T W I S T E D   M A S S E S                         %
%                                                                              %
%                                                                              %
%%%%%%%%%%%%%%%%%%%%%%%%%%%%%%%%%%%%%%%%%%%%%%%%%%%%%%%%%%%%%%%%%%%%%%%%%%%%%%%%
\section{Twisted Masses}
\label{stwist}
	
	Generalization to the theory with twisted masses is quite trivial. 
	In essense, every instance of $ \sigma $ in the Lagrangian is replaced
	with the difference $ \sigma - m_k $.
	Since all masses are generically different, the overall factor $ N/4\pi $ is
	replaced by $ 1/4\pi $ and a sum over $ k $.

	It is straightforward to write a superfield generalization of this.
	In Eq.~\eqref{Lsuper}, every occurence of superfield $ \Sigma $ 
	should be replaced with $ \Sigma - m_k $, and, in addition,
	we have to introduce $ N $ fields $ S_k $.
	It is of absolutely no effort to also include the heterotic deformation
	(for which, however, replacement $ \sigma \,\to\, \sigma - m_k $ is not done).
	The overall supersymmetric form of the effective Lagrangian with 
	twisted masses and heterotic deformation is,
\begin{align}
%
\notag
	& 4\pi\, \cell    ~~=~~     
			-\, \sum_k\, \Bigg\lgroup\,
			\int\, d^4\theta\, \frac{1}{2}\, \Big| \ln \big(\sqrt{2}\Sigma \,-\, m_k \big)\, \Big|^2
			~~+
	\\[2mm]
%
\notag
			&
			+~~
			i\, \int\, d^2\tilde\theta 
			\lgr
			\big( \sqrt{2}\Sigma \,-\, m_k \big)\, \ln \big( \sqrt{2}\Sigma \,-\, m_k \big)  ~-~ 
					\big( \sqrt{2}\Sigma \,-\, m_k \big)
			\rgr \!\!
			~~+
	\\[2mm]
%
\label{hetmass}
			&
			+~~ 
			\frac{1}{4}\, \int\, d^4\theta\, \ln\, \big( \sqrt{2}\ov\Sigma \,-\, \ov m{}_k \big)\, \times
	\\[1mm]
%
\notag
			&
			\qquad\qquad~
			\times
			\bigg\lgroup \Big( 1 \,+\, \frac{\sqrt{2}\Sigma \,-\, m_k}{S_k} \Big)^2\,
				\ln \left(\, 1 \,+\, \frac{S_k}{\sqrt{2}\Sigma \,-\, m_k} \,\right) ~-~
				\frac{\sqrt{2}\Sigma \,-\, m_k}{S_k} \bigg\rgroup \Bigg\rgroup \!\!
			~~+
	\\[2mm]
%
\notag
			&
			+~~
			4\pi\, \int\, d^2\theta_R\,\, \hzeta_R\, \hbzeta_R
			~~-~~
			4\pi i\, \int d\theta_R\,\, \hzeta\cdot J(\sqrt{2}\hsigma)
			~~+~~
			\text{h.c.}
\end{align}
	Hermitean conjugate is understood here to be added only to those terms that require it
	(in particular, the first term in Eq.~\eqref{hetmass} does not need a conjugate).
	We have introduced the obvious notation,
\beq
	S_k    ~~=~~    \frac{i}{2}\, \ov D{}_R D_L\, \ln \big( \sqrt{2}\ov\Sigma \,-\, \ov m{}_k \big)\,,
	\qquad\qquad
	\ov S{}_k    ~~=~~    \frac{i}{2}\, \ov D{}_L D_R\, \ln \big( \sqrt{2}\Sigma \,-\, m_k \big)\,.
\eeq




\pagebreak
%%%%%%%%%%%%%%%%%%%%%%%%%%%%%%%%%%%%%%%%%%%%%%%%%%%%%%%%%%%%%%%%%%%%%%%%%%%%%%%%
%                                                                              %
%                                                                              %
%                              C O N C L U S I O N                             %
%                                                                              %
%                                                                              %
%%%%%%%%%%%%%%%%%%%%%%%%%%%%%%%%%%%%%%%%%%%%%%%%%%%%%%%%%%%%%%%%%%%%%%%%%%%%%%%%
\section{Conclusions}
\label{sfinal}


%%%%%%%%%%%%%%%%%%%%%%%%%%%%%%%%%%%%%%%%%%%%%%%%%%%%%%%%%%%%%%%%%%%%%%%%%%%%%%%%
%                                                                              %
%                                                                              %
%                         A C K N O W L E D G M E N T S                        %
%                                                                              %
%                                                                              %
%%%%%%%%%%%%%%%%%%%%%%%%%%%%%%%%%%%%%%%%%%%%%%%%%%%%%%%%%%%%%%%%%%%%%%%%%%%%%%%%
\section*{Acknowledgments}

We are grateful to Sergey Gukov for stimulating questions  which motivated us to carry out this work.

	The work  of M.S. is supported in part by DOE grant DE-SC0011842. 
	The work of A.Y. was  supported 
	by  FTPI, University of Minnesota, 
	by RFBR Grant No. 13-02-00042a 
	and by Russian State Grant for 
	Scientific Schools RSGSS-657512010.2.




%%%%%%%%%%%%%%%%%%%%%%%%%%%%%%%%%%%%%%%%%%%%%%%%%%%%%%%%%%%%%%%%%%%%%%%%%%%%%%%%
%                                                                              %
%                                                                              %
%                              A P P E N D I C E S                             %
%                                                                              %
%                                                                              %
%%%%%%%%%%%%%%%%%%%%%%%%%%%%%%%%%%%%%%%%%%%%%%%%%%%%%%%%%%%%%%%%%%%%%%%%%%%%%%%%
\newpage
\appendix
\setcounter{equation}{0}


%%%%%%%%%%%%%%%%%%%%%%%%%%%%%%%%%%%%%%%%%%%%%%%%%%%%%%%%%%%%%%%%%%%%%%%%%%%%%%%%
%                                                                              %
%                               N O T A T I O N S                              %
%                                                                              %
%%%%%%%%%%%%%%%%%%%%%%%%%%%%%%%%%%%%%%%%%%%%%%%%%%%%%%%%%%%%%%%%%%%%%%%%%%%%%%%%
\section{Notations}
\label{app:notations}


%%%%%%%%%%%%%%%%%%%%%%%%%%%%%%%%%%%%%%%%%%%%%%%%%%%%%%%%%%%%%%%%%%%%%%%%%%%%%%%%
%                                                                              %
%                     T W I S T E D   S U P E R F I E L D S                    %
%                                                                              %
%%%%%%%%%%%%%%%%%%%%%%%%%%%%%%%%%%%%%%%%%%%%%%%%%%%%%%%%%%%%%%%%%%%%%%%%%%%%%%%%
\subsection{Twisted superfields}


	Twisted superfields exist in two dimensions and are defined by a ``twisted'' chirality condition
\beq
\label{gentwist}
	D_L\, \Sigma    ~~=~~    \ov D{}_R\, \Sigma    ~~=~~    0\,.
\eeq
	Analogously, for twisted anti-chiral superfields,
\beq
\label{genantitwist}
	D_R\, \ov\Sigma    ~~=~~    \ov D{}_L\, \ov\Sigma    ~~=~~    0\,.
\eeq
	The reason these conditions look so similar to those for chiral superfields is that
	there is no ideomatic difference between chiral and twisted-chiral superfields.
	To say more, they are interchanged by the action of mirror symmetry ---
	the transposition of supercharges turns condition \eqref{gentwist} into 
	the one for a chiral superfield.
	This way, as in the case with the chiral superfields, 
	the constraints \eqref{gentwist} and \eqref{genantitwist} are solved by letting
	the superfields be arbitrary functions of ``chiral'' variables $ \wt y^\mu $,
\begin{align*}
%%
	& \Sigma    ~~=~~    \Sigma(\wt y{}^\mu)
	&
	& \ov\Sigma    ~~=~~    \ov\Sigma(\ov{\wt y}{}^\mu)
	\\[2mm]
%%
	\wt y{}^0    ~~=~~ &    x^0  ~+~ i \left( \ov\theta{}_R\theta_R  ~-~  \ov\theta{}_L\theta_L \right)
	&
	\ov{\wt y}{}^0     ~~=~~ &    x^0  ~-~ i \left( \ov\theta{}_R\theta_R  ~-~  \ov\theta{}_L\theta_L \right)
	\\[2mm]
%%
	\wt y{}^3    ~~=~~ &    x^3  ~+~ \left( \ov\theta{}_R\theta_R  ~+~  \ov\theta{}_L\theta_L \right)
	&
	\ov{\wt y}{}^3    ~~=~~ &    x^3  ~-~ \left( \ov\theta{}_R\theta_R  ~+~  \ov\theta{}_L\theta_L \right)
	\,,
\end{align*}
	after which they will have the usual ``chiral'' component expansion
\begin{align*}
%%
	\Sigma(\wt y)    & ~~=~~    \sigma(\wt y)  ~~-~~  \sqrt{2}\, \theta_R \ov\lambda{}_L
						   ~~+~~  \sqrt{2}\, \ov\theta{}_L \lambda_R
						   ~~+~~  \sqrt{2}\, \theta_R \ov\theta{}_L\, \wt{F}
	\\[2mm]
%%
	\ov\Sigma(\ov{\wt y})    & ~~=~~    \ov\sigma(\ov{\wt y})  ~~-~~ \sqrt{2}\, \theta_L \ov\lambda{}_R
								   ~~+~~ \sqrt{2}\, \ov\theta{}_R \lambda_L
								   ~~+~~ \sqrt{2}\, \theta_L \ov\theta{}_R\, \ov{\wt F}
	\,.
\end{align*}
	Here we understand that each function on the right hand side depends on 
	$ \wt y{}^\mu $ and $ \ov{\wt y}{}^\mu $, correspondingly.

	Exactly the same way as with chiral superfields, one constructs {\it twisted superpotentials}
	$ \wt\cw(\Sigma) $, just as functions that depend on $ \Sigma $ holomorphically.
	One then performs the twisted $ d^2\tilde\theta $ integration as,
\beq
	\int\, d^2\tilde\theta\, \wt{\cw}(\Sigma)    ~~=~~    \frac{1}{2}\,\ov D{}_L\, D_R\, \wt{\cw}(\Sigma)\Big|\,,
	\qquad
	\int\, d^2\ov{\tilde\theta}\, \ov{\wt{\cw}}(\ov\Sigma)    ~~=~~    \frac{1}{2}\,\ov D{}_R\, D_L\, \ov{\wt{\cw}}(\ov\Sigma)\Big|\,.
\eeq
	And, of course, one can perform the full superspace integration of twisted superfields, provided
	that this holomorphicity is broken ({\it e.g.} by putting both chiral and anti-chiral factors),
\beq
	\int\, d^4\theta\, \ov\Sigma\, \Sigma\,,
\eeq
	or the result will obviously be a total derivative.


	One famous example of a twisted superfield is the {\it fieldstrength}
	of a \ntwot gauge supermultiplet $ V $,
\begin{align}
%%
	\Sigma    & ~~=~~    \frac{i}{\sqrt 2}\, D_L\, \ov D{}_R\, V\,,
	&
	\ov \Sigma    & ~~=~~    \frac{i}{\sqrt 2}\, D_R\, \ov D{}_L\, V\,.
\end{align}
	In components it takes the form
\begin{align*}
%%
	\Sigma(\wt y)    & ~~=~~    \sigma(\wt y)  ~~-~~  \sqrt{2}\, \theta_R \ov\lambda{}_L
						   ~~+~~  \sqrt{2}\, \ov\theta{}_L \lambda_R
						   ~~+~~  \sqrt{2}\, \theta_R \ov\theta{}_L \lgr D ~-~ i\, F_{03} \rgr
	\\[2mm]
%%
	\ov\Sigma(\ov{\wt y})    & ~~=~~    \ov\sigma(\ov{\wt y})  ~~-~~ \sqrt{2}\, \theta_L \ov\lambda{}_R
								   ~~+~~ \sqrt{2}\, \ov\theta{}_R \lambda_L
								   ~~+~~ \sqrt{2}\, \theta_L \ov\theta{}_R \lgr D ~+~ i\, F_{03} \rgr
	.
\end{align*}




%%%%%%%%%%%%%%%%%%%%%%%%%%%%%%%%%%%%%%%%%%%%%%%%%%%%%%%%%%%%%%%%%%%%%%%%%%%%%%%%
%                                                                              %
%                     N = ( 0 , 2 )   S U P E R F I E L D S                    %
%                                                                              %
%%%%%%%%%%%%%%%%%%%%%%%%%%%%%%%%%%%%%%%%%%%%%%%%%%%%%%%%%%%%%%%%%%%%%%%%%%%%%%%%
\subsection{\boldmath{\ntwoon} superfields}

	We define \ntwoo superspace via reduction of \ntwot superspace by putting
\beq
	\theta_L    ~~=~~    \ov\theta{}_L    ~~=~~    0\,.
\eeq
	Each chiral and twisted-chiral \ntwot superfield this way splits into two \ntwoo superfields.
	They still, however, retain their property of holomorphicity.
	While chiral superfields are usually described as being dependent on
	a ``holomorphic'' variable $ y^\mu $,
\beq
	y^\mu    ~~=~~    x^\mu    ~~+~~    i\,\ov{\theta\sigma}{}_\mu\theta\,,
\eeq
	and twisted-chiral as dependent on a twisted variable $ \wt y{}^\mu $,
\beq
	\wt y{}^\mu    ~~=~~    x^\mu    ~~+~~    i\,\ov{\theta\wt\sigma}{}_\mu\theta\,,
\eeq
	the distinction between the two variables vanishes upon reduction to \ntwoo superspace.
	As a result, \ntwoo superspace defines only one kind of holomorphic superfields --- 
	chiral \ntwoo superfields, which depend upon the reduced variables $ \upsilon^\mu $
\begin{align}
%
\notag
	\upsilon^0 &    ~~=~~    x^0  ~~+~~  i\,\ov\theta{}_R\theta_R
	\\[2mm]
%
	\upsilon^3 &    ~~=~~    x^3  ~~+~~  i\,\ov\theta{}_R\theta_R\,.
\end{align}

	To make a distinction with \ntwot superfields, we denote the \ntwoo superfields
	by symbols with a hat on top --- $ \hsigma $, $ \hxi $, {\it etc}.

	Chiral superfields $ \Phi(y) $ split into \ntwoo superfields 
	$ \hphi(\upsilon) $ and $ \hxi(\upsilon) $,
\begin{align}
%
\notag
	\Phi(y) &    ~~\longrightarrow~~    \hphi(\upsilon)  ~~+~~  \sqrt{2}\,\theta_L\,\hxi(\upsilon)\,,
	\\[2mm]
%
\label{Phisplit}
	\ov\Phi(\ov y) &    ~~\longrightarrow~~    \hbphi(\ov\upsilon)  ~~-~~  \sqrt{2}\,\ov\theta{}_L\,\hbxi(\ov\upsilon)\,,
\end{align}
	while twisted-chiral superfields $ \Sigma(\wt y) $ split into 
	$ \hsigma(\upsilon) $ and $ \hlambda(\upsilon) $,
\begin{align}
%
\notag
	\Sigma(\wt y) &    ~~\longrightarrow~~    \hsigma(\upsilon)  ~~+~~  \sqrt{2}\,\ov\theta{}_L\,\hlambda(\upsilon)\,,
	\\[2mm]
%
\label{Sigmasplit}
	\ov\Sigma(\ov{\wt y}) &    ~~\longrightarrow~~    \hbsigma(\ov\upsilon)  ~~-~~  \sqrt{2}\,\theta_L\,\hblambda(\upsilon)\,.
\end{align}
	We alert that relations \eqref{Phisplit} and \eqref{Sigmasplit} have only symbolical meaning demonstrating
	the effect of splitting, while there is no equality:
	the right-hand sides have incomplete dependence on $ \theta_L $, $ \ov\theta{}_L $.

	The individual \ntwoo superfields have a quite straightforward component expansion,
\begin{align}
%
\notag
	\hphi(\upsilon) &    ~~=~~    \phi  ~~-~~  \sqrt{2}\,\theta_R\,\psi_L\,,
	&
	\hxi(\upsilon) &    ~~=~~    \psi_R  ~~+~~  \sqrt{2}\,\theta_R\,F\,,
	\\[2mm]
%
	\hbphi(\ov\upsilon) &    ~~=~~    \ov \phi  ~~+~~  \sqrt{2}\,\ov{\theta_L \psi}{}_L\,,
	&
	\hbxi(\ov\upsilon) &    ~~=~~    \ov\psi{}_R  ~~+~~  \sqrt{2}\,\ov{\theta_R F}\,,
\end{align}
	and similarly do the ones that arise from splitting of the the twisted chiral superfield $ \Sigma(\wt y) $, 
\begin{align}
%
\notag
	\hsigma(\upsilon) &    ~~=~~    \sigma  ~~-~~  \sqrt{2}\,\theta_R\,\ov\lambda{}_L\,,
	&
	\hlambda(\upsilon) &    ~~=~~    \lambda_R  ~~-~~  \theta_R\,\wt F\,,
	\\[2mm]
%
	\hbsigma(\ov\upsilon) &    ~~=~~    \ov\sigma  ~~+~~  \sqrt{2}\,\ov\theta{}_R\,\lambda_L\,,
	&
	\hblambda(\ov\upsilon) &    ~~=~~    \ov\lambda{}_R  ~~-~~  \ov\theta{}_R\,\ov{\wt F}\,.
\end{align}

	
	It is interesting to note that the simple structure of splitting shown in \eqref{Phisplit} and \eqref{Sigmasplit}
	makes fermionic superfields in \ntwoo superspace much more ubiqutious than in \ntwot superspace.
	And the first example to this is the \ntwoo superpotential.
	It has to be fermionic because the integration over {\it half} of the \ntwoo superspace $ d\theta_R $ is such.
	A superpotential can be constructed using an arbitrary holomorphic function --- say $ J(\hsigma) $, and
	by multiplying it by an arbitrary (but still chiral) fermionic multiplet --- say $ \hat{\rho} $,
\beq
	\int\, d\theta_R\, \hat{\rho}\, J(\hsigma)\,.
\eeq

	``Full'' superspace integrals can conventionally be built using both chiral and antichiral fields,
\beq
	\int\, d^2\theta_R\,\, \hxi\, \hbxi    ~~=~~    \ov\psi{}_R\,i\p_L \psi_R  ~~+~~  \ov F\,F\,.
\eeq



%%%%%%%%%%%%%%%%%%%%%%%%%%%%%%%%%%%%%%%%%%%%%%%%%%%%%%%%%%%%%%%%%%%%%%%%%%%%%%%%
%                                                                              %
%                                                                              %
%           E X P A N S I O N   O F   E F F E C T I V E   A C T I O N          %
%                                                                              %
%                                                                              %
%%%%%%%%%%%%%%%%%%%%%%%%%%%%%%%%%%%%%%%%%%%%%%%%%%%%%%%%%%%%%%%%%%%%%%%%%%%%%%%%
\section{Component Expansion of the Effective Action}
\label{app:expansion}

	Here we give the complete component expansion of expression \eqref{Lsuper}. 
	Typically, one would be interested in a specific limit of this expression,
	such as the bosonic part of it, or the constant bosonic part, or 
	an approximation in the certain number of space-time derivatives
	(some approximations, however, are easier to derive from 
	the series representation \eqref{sseries}).

	We make a remark that, according to \cite{1p}, the whole fermionic part of
	the below effective Lagrangian can be obtained from its bosonic part simply
	via a replacement
\beq
	\sqrt{2}\sigma    ~~\longrightarrow~~    \sqrt{2}\sigma  ~~+~~  2\, \frac{i\,\ov\lambda{}_L\lambda_R}{iD + F_{03}}\,.
\eeq

	It is useful to know the lowest component of the superfield $ S $,
\begin{align*}
%
	s    ~~=~~
	S \Big| &    ~~=~~    \frac{ \sqrt{2}\ov\sigma\, ( iD \,-\, F_{03} ) ~-~ 2\,i\, \ov\lambda{}_R \lambda_L }
						{ \big( \sqrt{2}\, \ov\sigma \big)^2 }\,,
	\\[2mm]
%
	\ov s    ~~=~~
	\ov S \Big| &    ~~=~~    \frac{ \sqrt{2}\sigma\, ( iD \,+\, F_{03} ) ~-~ 2\,i\, \ov\lambda{}_L \lambda_R }
						{ \big( \sqrt{2}\, \sigma \big)^2 }\,.
\end{align*}
	In the expression below, however, we extensively make use of the lowest component of the ratio $ S / (\sqrt{2}\Sigma) $,
	which we denote as $ p $,
\begin{align*}
%
	p    ~~=~~
	\frac{S}{\sqrt{2}\Sigma} \bigg| &    ~~=~~
		\frac{1}{\big|\sqrt{2}\sigma\big|^2}
		\lgr
			iD \,-\, F_{03}
			~~-~~
			\frac{2\, i\sqrt{2}\sigma \ov\lambda{}_R\lambda_L}{\big|\sqrt{2}\sigma\big|^2}
		\rgr,
	\\[2mm]
%
	\ov p    ~~=~~
	\frac{\ov S}{\sqrt{2}\ov\Sigma} \bigg| &    ~~=~~
		\frac{1}{\big|\sqrt{2}\sigma\big|^2}
		\lgr
			iD \,+\, F_{03}
			~~-~~
			\frac{2\, i\sqrt{2}\ov{\sigma\lambda}{}_L\lambda_R}{\big|\sqrt{2}\sigma\big|^2}
		\rgr.
\end{align*}

	
	We have,
\begingroup
\allowdisplaybreaks
\begin{align}
%
\notag
	& \frac{4\pi}{N}\, \cell    ~~=~~     
			iD  ~~-~~  F_{03}\, \log\, \frac{\sqrt{2}\sigma}{\sqrt{2}\ov\sigma}
		~~-~~ i\, \frac{ \sqrt{2}\sigma\ov\lambda{}_R\lambda_L ~+~ \sqrt{2}\ov{\sigma\lambda}{}_L\lambda_R }
				{ \big|\sqrt{2}\sigma\big|^2 }
		~~-
	\\[2mm]
%
\notag
	&
		~~-~~  \frac{1}{4}\, \ln \sqrt{2}\sigma\, \Box\, \ln \sqrt{2}\sigma
		~~-~~  \frac{1}{4}\, \ln \sqrt{2}\ov\sigma\, \Box\, \ln \sqrt{2}\ov\sigma
		~~-
	\\[2mm]
%
\notag
	&
		~~-~~  \frac{1}{2} \Big\lgroup iD  \,+\,  F_{03}  \,+\,  \frac{1}{2}\,\Box\,\ln \sqrt{2}\ov\sigma \Big\rgroup
			\ln \lgr \big|\sqrt{2}\sigma\big|^2  \,+\,  iD  \,-\,  F_{03}  
			\,-\,  2i\, \frac{\ov\lambda{}_R\lambda_L}{\sqrt{2}\ov\sigma} \rgr 
		~-
	\\[2mm]
%
\notag
	&
		~~-~~  \frac{1}{2} \Big\lgroup iD  \,-\,  F_{03}  \,+\,  \frac{1}{2}\,\Box\,\ln \sqrt{2}\sigma \Big\rgroup
			\ln \lgr \big|\sqrt{2}\sigma\big|^2  \,+\,  iD  \,+\,  F_{03}
			\,-\,  2i\, \frac{\ov\lambda{}_L\lambda_R}{\sqrt{2}\sigma} \rgr
	\\[2mm]
%
\notag
	&
		~~+~~  
			\frac{ 
				-\, \ov\lambda{}_R\, i\overleftarrow{\md}_L \lambda_R\,   \,+\, \ov{\lambda}{}_L\, i\md_R \lambda_L \,-\,
				\frac{\displaystyle 1}{\displaystyle 2}\, i\,\sqrt{2}\ov{\sigma \lambda}{}_L \lambda_R
				\,+\, \frac{\displaystyle 1}{\displaystyle 2}\,
					\frac{\displaystyle 1}{\displaystyle\sqrt{2}\ov\sigma}\, 
					i\, \ov\lambda{}_R \overleftarrow{\md}{}_L\, \md_R \lambda_L
			}
			{
				\big|\sqrt{2}\sigma\big|^2 \,+\, iD \,-\, F_{03} 
				\,-\, 2i\,\frac{\displaystyle \ov\lambda{}_R\lambda_L}{\displaystyle \sqrt{2}\ov\sigma}
			}
	\\[2mm]
%
\notag
	&
		~~+~~  
			\frac{
				\ov\lambda{}_R\, i\md_L\lambda_R \,-\, \ov\lambda{}_L\, i\overleftarrow{\md}_R \lambda_L \,-\,
				\frac{\displaystyle 1}{\displaystyle 2}\, i\,\sqrt{2}\sigma \ov\lambda{}_R \lambda_L
				\,+\, \frac{\displaystyle 1}{\displaystyle 2}\,
					\frac{\displaystyle 1}{\displaystyle\sqrt{2}\sigma}\,
					i\, \ov\lambda{}_L \overleftarrow{\md}_R\, \md_L\lambda_R
			}
			{
				\big|\sqrt{2}\sigma\big|^2 \,+\, iD \,+\, F_{03} 
				\,-\, 2i\,\frac{\displaystyle \ov\lambda{}_L\lambda_R}{\displaystyle \sqrt{2}\sigma}
			}
	\\[2mm]
%
\notag
	&
		~~-~~  \frac{1}{4}\,
			\frac{
				\big|\sqrt{2}\sigma\big|^2
				\lgr iD \,+\, F_{03} \,+\, \Box\, \ln \sqrt{2}\ov\sigma \rgr
			}
			{
				\big|\sqrt{2}\sigma\big|^2 \,+\, iD \,-\, F_{03} 
				\,-\, 2i\,\frac{\displaystyle \ov\lambda{}_R\lambda_L}{\displaystyle \sqrt{2}\ov\sigma}
			}
		~~-~~  \frac{1}{4}\,
			\frac{
				\big|\sqrt{2}\sigma\big|^2
				\lgr iD \,-\, F_{03} \,+\, \Box\, \ln \sqrt{2}\sigma \rgr
			}
			{
				\big|\sqrt{2}\sigma\big|^2 \,+\, iD \,+\, F_{03} 
				\,-\, 2i\,\frac{\displaystyle \ov\lambda{}_L\lambda_R}{\displaystyle \sqrt{2}\sigma}
			}
	\\[2mm]
%
\notag
	&
		~~-~~  \frac{\big|\sqrt{2}\sigma\big|^2}{2}
			\lgr 1 \,+\, \frac{1}{p} \rgr \!
			\frac{ 
				-\, \ov\lambda{}_R\, i\overleftarrow{\md}_L \lambda_R  \,+\, \ov{\lambda}{}_L\, i\md_R \lambda_L \,+\,
				i\,\sqrt{2}\ov{\sigma \lambda}{}_L \lambda_R
				\,+\, \frac{\displaystyle 1}{\displaystyle \sqrt{2}\ov\sigma}\,
					i\, \ov\lambda{}_R \overleftarrow{\md}_L\, \md_R\lambda_L
			}
			{
			\lgr  
				\big|\sqrt{2}\sigma\big|^2 \,+\, iD \,-\, F_{03} 
				\,-\, 2i\,\frac{\displaystyle \ov\lambda{}_R\lambda_L}{\displaystyle \sqrt{2}\ov\sigma}  
			\rgr^2
			}
	\\[2mm]
%
\notag
	&
		~~-~~  \frac{\big|\sqrt{2}\sigma\big|^2}{2}
			\lgr 1 \,+\, \frac{1}{\ov p} \rgr\!
			\frac{
				\ov\lambda{}_R\, i\md_L\lambda_R \,-\, \ov\lambda{}_L\, i\overleftarrow{\md}_R \lambda_L \,+\,
				i\,\sqrt{2}\sigma \ov\lambda{}_R \lambda_L
				\,+\, \frac{\displaystyle 1}{\displaystyle \sqrt{2}\sigma}\,
					i\, \ov\lambda{}_L \overleftarrow{\md}_R\, \md_L\lambda_R
			}
			{
			\lgr
				\big|\sqrt{2}\sigma\big|^2 \,+\, iD \,+\, F_{03} 
				\,-\, 2i\,\frac{\displaystyle \ov\lambda{}_L\lambda_R}{\displaystyle \sqrt{2}\sigma}
			\rgr^2
			}
	\\[2mm]
%
	&
		~~+~~  \frac{1}{2\, p^3}
			\Bigg\lgroup
				-\, p^2\, \big( iD \,+\ F_{03} \big)
				~+~  \frac{1}{2}\, p\, \Box \ln \sqrt{2}\ov\sigma
				~-~  2\, p^2\, i\, \frac{\ov\lambda{}_L \lambda_R}{\sqrt{2}\sigma}  ~+~
	\\[2mm]
%
\notag
	&
		\phantom{~~+~~  \frac{1}{2\, p^3} \Bigg\lgroup}
				+\, 2\, i\, \frac{1}{\sqrt{2}\sigma}\, 
					\frac{\ov\lambda{}_R \overleftarrow{\md}_L\, \md_R\lambda_L}{\big(\sqrt{2}\ov\sigma\big)^2}
				~-~  2\, p\, 
					\frac{\ov\lambda{}_L\, i\md_R \lambda_L ~-~ \ov\lambda{}_R\, i\overleftarrow{\md}_L \lambda_R}{\big|\sqrt{2}\sigma\big|^2}
			\Bigg\rgroup\!\!
		\cdot \ln \big( 1 ~+~ p \big)
	\\[2mm]
%
\notag
	&
		~~+~~  \frac{1}{2\, \ov p{}^3}
			\Bigg\lgroup
				-\, \ov p{}^2\, \big( iD \,-\, F_{03} \big)
				~+~  \frac{1}{2}\, \ov p\, \Box \ln \sqrt{2}\sigma
				~-~  2\, \ov p{}^2\, i\, \frac{\ov\lambda{}_R \lambda_L}{\sqrt{2}\ov\sigma}  ~+~
	\\[2mm]
%
\notag
	&
		\phantom{~~+~~  \frac{1}{2\, \ov p{}^3} \Bigg\lgroup}
				+\, 2\, i\, \frac{1}{\sqrt{2}\ov\sigma}\,
					\frac{\ov\lambda{}_L \overleftarrow{\md}_R\, \md_L\lambda_L}{\big(\sqrt{2}\sigma\big)^2}
				~-~  2\, \ov p\,
					\frac{\ov\lambda{}_R\, i\md_L \lambda_R ~-~ \ov\lambda{}_L\, i\overleftarrow{\md}_R \lambda_L}{\big|\sqrt{2}\sigma\big|^2}
			\Bigg\rgroup\!\!
		\cdot \ln \big( 1 ~+~ \ov p \big)
	\\[2mm]
%
\notag
	&
		~~+~~  \frac{1}{2\, p^2}\, i
		\lgr
			\frac{
				-\, \sqrt{2}\ov{\sigma\lambda}{}_L 
				\,-\, \ov\lambda{}_R \overleftarrow{\md}_L
			}
			{
				\big|\sqrt{2}\sigma\big|^2 \,+\, iD \,-\, F_{03} 
				\,-\, 2i\,\frac{\displaystyle \ov\lambda{}_R\lambda_L}{\displaystyle \sqrt{2}\ov\sigma}
			}
			~+~
			\frac{\ov\lambda{}_L}{\sqrt{2}\sigma}
		\rgr
		\!\!
		\lgr
			2\,p\,\lambda_R
			~+~
			\frac{\md_R \lambda_L}{\sqrt{2}\ov\sigma}
		\rgr
	\\[2mm]
%
\notag
	&
		~~+~~  \frac{1}{2\, \ov p{}^2}\, i
		\lgr
			\frac{
				-\, \sqrt{2}\sigma\ov\lambda{}_R
				\,-\, \ov\lambda{}_L \overleftarrow{\md}_R
			}
			{
				\big|\sqrt{2}\sigma\big|^2 \,+\, iD \,+\, F_{03} 
				\,-\, 2i\,\frac{\displaystyle \ov\lambda{}_L\lambda_R}{\displaystyle \sqrt{2}\sigma}
			}
			~+~
			\frac{\ov\lambda{}_R}{\sqrt{2}\ov\sigma}
		\rgr
		\!\!
		\lgr
			2\, \ov p\, \lambda_L
			~+~
			\frac{\md_L \lambda_R}{\sqrt{2}\sigma}
		\rgr
	\\[2mm]
%
\notag
	&
		~~+~~  \frac{1}{2\, p^2}\, i
		\lgr
			-\, 2\, p\, \ov\lambda{}_L
			~+~
			\frac{\ov\lambda{}_R \overleftarrow{\md}_L}{\sqrt{2}\ov\sigma}
		\rgr
		\!\!
		\lgr
			\frac{
				\sqrt{2}\ov\sigma\lambda_R
				\,-\, \md_R \lambda_L
			}
			{
				\big|\sqrt{2}\sigma\big|^2 \,+\, iD \,-\, F_{03} 
				\,-\, 2i\,\frac{\displaystyle \ov\lambda{}_R\lambda_L}{\displaystyle \sqrt{2}\ov\sigma}
			}
			~-~
			\frac{\lambda_R}{\sqrt{2}\sigma}
		\rgr
	\\[2mm]
%
\notag
	&
		~~+~~  \frac{1}{2\, \ov p{}^2}\, i
		\lgr
			-\, 2\, \ov p\, \ov\lambda{}_R
			~+~
			\frac{\ov\lambda{}_L \overleftarrow{\md}_R}{\sqrt{2}\sigma}
		\rgr
		\!\!
		\lgr
			\frac{
				\sqrt{2}\sigma\lambda_L
				\,-\, \md_L\lambda_R
			}
			{
				\big|\sqrt{2}\sigma\big|^2 \,+\, iD \,+\, F_{03} 
				\,-\, 2i\,\frac{\displaystyle \ov\lambda{}_L\lambda_R}{\displaystyle \sqrt{2}\sigma}
			}
			~-~
			\frac{\lambda_L}{\sqrt{2}\ov\sigma}
		\rgr
	\\[2mm]
%
\notag
	&
		~~-~~  \frac{1}{4\, p}\, \big|\sqrt{2}\sigma\big|^2\,
			\frac{iD \,+\, F_{03} \,+\, \Box \ln \sqrt{2}\ov\sigma}
			{
				\big|\sqrt{2}\sigma\big|^2 \,+\, iD \,-\, F_{03} 
				\,-\, 2i\,\frac{\displaystyle \ov\lambda{}_R\lambda_L}{\displaystyle \sqrt{2}\ov\sigma}
			}
		~~+~~  \frac{1}{4}\, \big|\sqrt{2}\sigma\big|^2\, \frac{\ov p}{p}
	\\[2mm]
%
\notag
	&
		~~-~~  \frac{1}{4\, \ov p}\, \big|\sqrt{2}\sigma\big|^2\,
			\frac{iD \,-\, F_{03} \,+\, \Box \ln \sqrt{2}\sigma}
			{
				\big|\sqrt{2}\sigma\big|^2 \,+\, iD \,+\, F_{03} 
				\,-\, 2i\,\frac{\displaystyle \ov\lambda{}_L\lambda_R}{\displaystyle \sqrt{2}\sigma}
			}
		~~+~~  \frac{1}{4}\, \big|\sqrt{2}\sigma\big|^2\, \frac{p}{\ov p}
	\,.
\end{align}
\endgroup




%%%%%%%%%%%%%%%%%%%%%%%%%%%%%%%%%%%%%%%%%%%%%%%%%%%%%%%%%%%%%%%%%%%%%%%%%%%%%%%%
%                                                                              %
%                                                                              %
%                            B I B L I O G R A P H Y                           %
%                                                                              %
%                                                                              %
%%%%%%%%%%%%%%%%%%%%%%%%%%%%%%%%%%%%%%%%%%%%%%%%%%%%%%%%%%%%%%%%%%%%%%%%%%%%%%%%
\small
\begin{thebibliography}{99}

  \bibitem{0}
E.~Witten,
 {\em Instantons, the Quark Model, and the 1/n Expansion,}
  Nucl.\ Phys.\ B {\bf 149}, 285 (1979).
  %%CITATION = NUPHA,B149,285;%%
  %678 citations counted in INSPIRE as of 15 Aug 2014
  
   \bibitem{1p}
   A.~D'Adda, P.~Di Vecchia and M.~Luscher,
{\em Confinement and Chiral Symmetry Breaking in CP(N-1) Models with Quarks,}
  Nucl.\ Phys.\ B {\bf 152}, 125 (1979).
  %%CITATION = NUPHA,B152,125;%%
  %354 citations counted in INSPIRE as of 18 Aug 2014
  
    \bibitem{SYhet}
    M.~Shifman and A.~Yung,
{\em Large-N Solution of the Heterotic N=(0,2) Two-Dimensional CP(N-1) Model,}
  Phys.\ Rev.\  D {\bf 77}, 125017 (2008)
  [arXiv:0803.0698 [hep-th]].
  %%CITATION = PHRVA,D77,125017;%%

  \bibitem{1}
  A.~D'Adda, A.~C.~Davis, P.~Di Vecchia and P.~Salomonson,
 {\em An Effective Action for the Supersymmetric {CP}${(N-1)}$ Model,}
  Nucl.\ Phys.\ B {\bf 222}, 45 (1983).
  %%CITATION = NUPHA,B222,45;%%
  %63 citations counted in INSPIRE as of 15 Aug 2014
  
    \bibitem{2}
    S.~Cecotti and C.~Vafa,
  {\em On classification of ${\mathcal N}=2$ supersymmetric theories,}
  Commun.\ Math.\ Phys.\  {\bf 158}, 569 (1993)
  [hep-th/9211097].
  %%CITATION = HEP-TH/9211097;%%
  %206 citations counted in INSPIRE as of 15 Aug 2014
  
      \bibitem{3}
      E.~Witten,
{\em Phases of ${\mathcal N}=2$ theories in two-dimensions,}
  Nucl.\ Phys.\ B {\bf 403}, 159 (1993)
  [hep-th/9301042].
  %%CITATION = HEP-TH/9301042;%%
  %898 citations counted in INSPIRE as of 15 Aug 2014
  
%%  
  
  \bibitem{Veneziano}
  G.~Veneziano and S.~Yankielowicz,
  {\em An Effective Lagrangian for the Pure ${\mathcal N}=1$ Supersymmetric Yang-Mills Theory,}
  Phys.\ Lett.\ B {\bf 113} (1982) 231.
  %%CITATION = PHLTA,B113,231;%%
  %645 citations counted in INSPIRE as of 18 Aug 2014
  
   \bibitem{BSYhet}
  P.~A.~Bolokhov, M.~Shifman and A.~Yung,
  {\em Large-N Solution of the Heterotic CP(N-1) Model with Twisted Masses,}
  Phys.\ Rev.\ D {\bf 82}, 025011 (2010)
  [arXiv:1001.1757 [hep-th]].
  %%CITATION = ARXIV:1001.1757;%%
  %9 citations counted in INSPIRE as of 18 Aug 2014

  \bibitem{EdTo}
   M.~Edalati and D.~Tong,
 {\em Heterotic vortex strings,}
  JHEP {\bf 0705}, 005 (2007)
  [arXiv:hep-th/0703045].
  %%CITATION = JHEPA,0705,005;%%
  
  \bibitem{SY1}
  M.~Shifman and A.~Yung,
  {\em Heterotic Flux Tubes in ${\mathcal N}=2$ SQCD with ${\mathcal N}=1$ Preserving Deformations,}
  Phys.\ Rev.\  D {\bf 77}, 125016 (2008)
  [arXiv:0803.0158 [hep-th]].
  %%CITATION = PHRVA,D77,125016;%%

  \bibitem{BSY1}
  P.~A.~Bolokhov, M.~Shifman and A.~Yung,
  {\em Description of the Heterotic String Solutions in U(N) SQCD,}
  Phys. \ Rev. \ D {\bf 79}, 085015 (2009) (Erratum: Phys. Rev. D 80, 049902 (2009))
  [arXiv:0901.4603 [hep-th]].
  %%CITATION = ARXIV:0901.4603;%%
%%  
%%  \bibitem{BSY2}
%%  P.~A.~Bolokhov, M.~Shifman and A.~Yung,
%%  %``Description of the Heterotic String Solutions in the M Model,''
%%  Phys. \ Rev. \ D {\bf 79}, 106001 (2009) (Erratum: Phys. Rev. D 80, 049903 (2009))
%%  [arXiv:0903.1089 [hep-th]].
%%  %%CITATION = ARXIV:0903.1089;%%  
%%  

  \bibitem{BSY3}
  P.~A.~Bolokhov, M.~Shifman and A.~Yung,
  %``Heterotic N=(0,2) CP(N-1) Model with Twisted Masses,''
  Phys.\ Rev.\  D {\bf 81}, 065025 (2010)
  [arXiv:0907.2715 [hep-th]].
  %%CITATION = PHRVA,D81,065025;%%

%%  
%%  \bibitem{orco}
%%  E.~Witten,
%%  %``A Supersymmetric Form Of The Nonlinear Sigma Model In Two-Dimensions,''
%%  Phys.\ Rev.\  D {\bf 16}, 2991 (1977);
%%  %%CITATION = PHRVA,D16,2991;%%
%%  P.~Di Vecchia and S.~Ferrara,
%%  %``Classical Solutions In Two-Dimensional Supersymmetric Field Theories,''
%%  Nucl.\ Phys.\  B {\bf 130}, 93 (1977).
%%  %%CITATION = NUPHA,B130,93;%%
%%
%%  \bibitem{Bruno}
%%  B.~Zumino,
%%  %``Supersymmetry And Kahler Manifolds,''
%%  Phys.\ Lett.\  B {\bf 87}, 203 (1979).
%%  %%CITATION = PHLTA,B87,203;%%
%%
%%  \bibitem{rev1}
%%  V.~A.~Novikov, M.~A.~Shifman, A.~I.~Vainshtein and V.~I.~Zakharov,
%%  %``Two-Dimensional Sigma Models: Modeling Nonperturbative Effects Of Quantum
%%  %Chromodynamics,''
%%  Phys.\ Rept.\  {\bf 116}, 103 (1984).
%%  %%CITATION = PRPLC,116,103;%%
%%   
%%  \bibitem{rev2} 
%%  A.~M.~Perelomov,
%%  %``SUPERSYMMETRIC CHIRAL MODELS: GEOMETRICAL ASPECTS,''
%%  Phys.\ Rept.\  {\bf 174}, 229 (1989).
%%  %%CITATION = PRPLC,174,229;%%
%%  
%%  \bibitem{WI}
%%  E.~Witten,
%%  %``Constraints On Supersymmetry Breaking,''
%%  Nucl.\ Phys.\  B {\bf 202}, 253 (1982).
%%  %%CITATION = NUPHA,B202,253;%%
%%  
%%  \bibitem{twisted}
%%  L.~Alvarez-Gaum\'{e} and D.~Z.~Freedman,
%%  %``Potentials For The Supersymmetric Nonlinear Sigma Model,''
%%  Commun.\ Math.\ Phys.\  {\bf 91}, 87 (1983);
%%  %%CITATION = CMPHA,91,87;%%
%%  S.~J.~Gates,
%%  %``Superspace Formulation Of New Nonlinear Sigma Models,''
%%  Nucl.\ Phys.\ B {\bf 238}, 349 (1984);
%%  %%CITATION = NUPHA,B238,349;%%
%%  S.~J.~Gates, C.~M.~Hull and M.~Ro\v{c}ek,
%%  %``Twisted Multiplets And New Supersymmetric Nonlinear Sigma Models,''
%%  Nucl.\ Phys.\ B {\bf 248}, 157 (1984).
%%  %%CITATION = NUPHA,B248,157;%%
%%
%%  \bibitem{BelPo}
%%  A.~M.~Polyakov,
%%  %``Interaction Of Goldstone Particles In Two-Dimensions. Applications To
%%  %Ferromagnets And Massive Yang-Mills Fields,''
%%  Phys.\ Lett.\  B {\bf 59}, 79 (1975).
%%  %%CITATION = PHLTA,B59,79;%%
%%  
%%  \bibitem{adam}
%%  A.~Ritz, M.~Shifman and A.~Vainshtein,
%%  %``Counting domain walls in N = 1 super Yang-Mills,''
%%  Phys.\ Rev.\  D {\bf 66}, 065015 (2002)
%%  [arXiv:hep-th/0205083].
%%  %%CITATION = PHRVA,D66,065015;%%
%%  
%%  \bibitem{WessBagger}
%%  J. Wess and J. Bagger, {\em Supersymmetry and Supergravity}, Second Edition,
%%  Princeton University Press, 1992.
%%
%%  \bibitem{Helgason}
%%  S. Helgason, {\sl Differential geometry, Lie groups and symmetric spaces},
%%  Academic Press, New York, 1978.
%%  
%%  \bibitem{Dor}
%%  N.~Dorey,
%%  %``The BPS spectra of two-dimensional
%%  %supersymmetric gauge theories
%%  %with  twisted mass terms,''
%%  JHEP {\bf 9811}, 005 (1998) [hep-th/9806056].
%%  %%CITATION = HEP-TH 9806056;%%
%%
%%  \bibitem{Witten:2005px}
%%  E.~Witten,
%%  {\em Two-dimensional models with (0,2) supersymmetry: Perturbative aspects,}
%%  arXiv:hep-th/0504078.
%%  %%CITATION = HEP-TH/0504078;%%
%%  
%%  \bibitem{GSYphtr}
%%  A.~Gorsky, M.~Shifman and A.~Yung,
%%   %``Higgs and Coulomb/confining phases in "twisted-mass" deformed \cpn model,''
%%  Phys.\ Rev.\ D {\bf 73}, 065011 (2006)
%%  [hep-th/0512153].
%%
%%  \bibitem{Coleman}
%%  S.~R.~Coleman,
%%  %``More About The Massive Schwinger Model,''
%%  Annals Phys.  {\bf 101}, 239 (1976).
%%
%%  \bibitem{GSY05}
%%  A.~Gorsky, M.~Shifman and A.~Yung,
%%   %``Non-Abelian Meissner effect in Yang-Mills theories at weak
%%  %coupling,''
%%  Phys.\ Rev.\ D {\bf 71}, 045010 (2005)
%%  [hep-th/0412082].
%%  %%CITATION = HEP-TH 0412082;%%
%%
%%  \bibitem{Ferrari}
%%  F.~Ferrari,
%%  % ``LARGE N AND DOUBLE SCALING LIMITS IN TWO-DIMENSIONS.''
%%  JHEP {\bf 0205} 044 (2002)
%%  [hep-th/0202002].
%%
%%  \bibitem{Ferrari2}
%%  F.~Ferrari,
%%  %'' NONSUPERSYMMETRIC COUSINS OF SUPERSYMMETRIC GAUGE THEORIES:
%%  %QUANTUM SPACE OF PARAMETERS AND DOUBLE SCALING LIMITS.''
%%   Phys. Lett. {\bf B496} 212 (2000)
%%  [hep-th/0003142];
%%  %``A model for gauge theories with Higgs fields,''
%%  JHEP {\bf 0106}, 057 (2001)
%%  [hep-th/0102041].
%%  %%CITATION = HEP-TH 0102041;%%
%%
%%  \bibitem{AdDVecSal}
%%  A.~D'Adda, A.~C.~Davis, P.~DiVeccia and P.~Salamonson,
%%  %"An effective action for the supersymmetric CP$^{n-1}$ models,"
%%  Nucl.\ Phys.\ {\bf B222} 45 (1983).
%%
%%  \bibitem{ChVa}
%%  S.~Cecotti and C. Vafa,
%%  %"On classification of \ntwo supersymmetric theories,"
%%  Comm. \ Math. \ Phys. \ {\bf 158} 569 (1993)
%%  [hep-th/9211097].
%%
%%  \bibitem{HaHo}
%%  A.~Hanany and K.~Hori,
%%  %``Branes and N = 2 theories in two dimensions,''
%%  Nucl.\ Phys.\  B {\bf 513}, 119 (1998)
%%  [arXiv:hep-th/9707192].
%%  %%CITATION = NUPHA,B513,119;%%
%%
%%  \bibitem{AD}
%%  P. C.~Argyres and M. R.~Douglas,
%%  %``New Phenomena in SU(3) Supersymmetric Gauge Theory'' 
%%  Nucl. \ Phys. \ {\bf B448}, 93 (1995)   
%%  [arXiv:hep-th/9505062].
%%  %%CITATION = NUPHA,B448,93;%%
%%  
%%  \bibitem{APSW}
%%  P. C. Argyres, M. R. Plesser, N. Seiberg, and E. Witten,
%%  %``New N=2 Superconformal Field Theories in Four Dimensions''
%%  Nucl. \ Phys.  \ {\bf B461}, 71 (1996) 
%%  [arXiv:hep-th/9511154].
%%  %%CITATION = NUPHA,B461,71;%%
%%
%%  \bibitem{SYrev}
%%  M.~Shifman and A.~Yung,
%%  %{\sl Supersymmetric Solitons,}
%%  Rev.\ Mod.\ Phys. {\bf 79} 1139 (2007)
%%  [arXiv:hep-th/0703267].
%%  %%CITATION = HEP-TH/0703267;%%
%%
%%  \bibitem{Tonghetdyn}
%%  D.~Tong,
%%  %``The quantum dynamics of heterotic vortex strings,''
%%  JHEP {\bf 0709}, 022 (2007)
%%  [arXiv:hep-th/0703235].
%%  %%CITATION = JHEPA,0709,022;%%
%%  
%%  \bibitem{VYan}
%%  G.~Veneziano and S.~Yankielowicz,
%%  %``An Effective Lagrangian For The Pure N=1 Supersymmetric Yang-Mills
%%  %Theory,''
%%  Phys.\ Lett.\  B {\bf 113}, 231 (1982).
%%  %%CITATION = PHLTA,B113,231;%%
%%  
%%  \bibitem{ls}
%%  A.~Losev and M.~Shifman,
%%  %``N = 2 sigma model with twisted mass and superpotential: Central charges
%%  %and solitons,''
%%  Phys.\ Rev.\  D {\bf 68}, 045006 (2003)
%%  [arXiv:hep-th/0304003].
%%  %%CITATION = PHRVA,D68,045006;%%
%%  
%%  \bibitem{ls1}
%%  M.~Shifman, A.~Vainshtein and R.~Zwicky,
%%  %``Central charge anomalies in 2D sigma models with twisted mass,''
%%  J.\ Phys.\ A  {\bf 39}, 13005 (2006)
%%  [arXiv:hep-th/0602004].
%%  %%CITATION = JPAGB,A39,13005;%
%%  
%%  \bibitem{SYneww}
%%  M.~Shifman and A.~Yung,
%%  %``N=(0,2) Deformation of the N=(2,2) Wess-Zumino Model in Two Dimensions,''
%%  Phys.\ Rev.\  D {\bf 81}, 105022 (2010)
%%  [arXiv:0912.3836 [hep-th]].
%%  %%CITATION = PHRVA,D81,105022;%%
%%  
%%  \bibitem{D1}
%%  J.~Distler and S.~Kachru,
%%  %``(0,2) Landau-Ginzburg theory,''
%%  Nucl.\ Phys.\  B {\bf 413}, 213 (1994)
%%  [arXiv:hep-th/9309110].
%%  %%CITATION = NUPHA,B413,213;%%
%%  
%%  \bibitem{D2}
%%  T.~Kawai and K.~Mohri,
%%  %``Geometry Of (0,2) Landau-Ginzburg Orbifolds,''
%%  Nucl.\ Phys.\  B {\bf 425}, 191 (1994)
%%  [arXiv:hep-th/9402148].
%%  %%CITATION = NUPHA,B425,191;%%
%%  
%%  \bibitem{D3}
%%  I.~V.~Melnikov,
%%  %``(0,2) Landau-Ginzburg Models and Residues,''
%%  JHEP {\bf 0909}, 118 (2009)
%%  [arXiv:0902.3908 [hep-th]].
%%  %%CITATION = JHEPA,0909,118;%%
%%  
%%  \bibitem{Shifman:2009ay}
%%  M.~Shifman and A.~Yung,
%%  %``Crossover between Abelian and non-Abelian confinement in N=2 supersymmetric
%%  %QCD,''
%%  Phys.\ Rev.\  D {\bf 79}, 105006 (2009)
%%  [arXiv:0901.4144 [hep-th]].
%%  %%CITATION = PHRVA,D79,105006;%%
%%
%%  \bibitem{Tadpoint}
%%  D.~Tong,
%%  %``Superconformal  vortex strings,''
%%  JHEP {\bf 0612}, 051 (2006)
%%  [arXiv:hep-th/0610214].
%%
%%  \bibitem{SMMS}
%%  A.~Migdal and M.~Shifman,
%%  %``Dilaton Effective Lagrangian In Gluodynamics,''
%%  Phys.\ Lett.\  B {\bf 114}, 445 (1982).
%%  %%CITATION = PHLTA,B114,445;%%
%%  
%%  \bibitem{Kos}
%%  A.~Kovner and M.~A.~Shifman,
%%  %``Chirally symmetric phase of supersymmetric gluodynamics,''
%%  Phys.\ Rev.\  D {\bf 56}, 2396 (1997)
%%  [arXiv:hep-th/9702174].
%%  %%CITATION = PHRVA,D56,2396;%%
%%
\end{thebibliography}

\end{document}
